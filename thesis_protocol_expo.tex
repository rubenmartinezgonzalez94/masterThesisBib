\documentclass[10pt,letterpaper,final]{article}
\usepackage[left=4cm,rigth=4cm,top=4cm,bottom=2cm]{geometry}
\usepackage[utf8]{inputenc}
\usepackage{amsmath}
\usepackage{amsfonts}
\usepackage{amssymb}
\usepackage{graphicx}
\usepackage{kpfonts}
\usepackage{tabularx}
\usepackage{hyperref}
\usepackage{natbib}
\begin{document}
    \section*{}
    \title{Sistema de Detección y Evasión de Colisiones basado en Visión Computacional para Vehículos Autónomos}
    \author{Rubén Martínez González}
    \maketitle
    \clearpage
    \section*{Introducción}
    \noindent
    El avance continuo en la tecnología de vehículos autónomos representa un logro significativo en la revolución del transporte.
    Existe la necesidad de desarrollar sistemas ‘inteligentes' que permitan a estos vehículos aprender a conducir de manera autónoma y,
    al mismo tiempo, detectar posibles colisiones y reaccionar de manera similar a como lo haría un conductor humano.
    La convergencia de la inteligencia artificial, la visión computacional y los sistemas de control ha generado una nueva era en la movilidad,
    desafiando y redefiniendo las fronteras de la conducción convencional.\\ \newline
    En este contexto, este trabajo se centra en el desarrollo de un sistema de detección y evasión de colisiones basado en visión computacional.
    Si bien los avances en la conducción autónoma han sido significativos, la detección y respuesta a
    situaciones de peligro, como colisiones inminentes, siguen siendo un desafío complejo.\\ \newline
    La seguridad en la carretera y la confianza del público en esta tecnología dependen en gran medida de la capacidad
    de los vehículos autónomos para enfrentar situaciones de tráfico de manera eficiente y segura.\\ \newline
    Esta investigación busca abordar esta problemática crítica, avanzando hacia un futuro en el que los vehículos autónomos
    sean capaces de igualar e incluso superar las habilidades de conducción humana en términos de detección y respuesta
    a situaciones de colisión.
    \clearpage
    
    \section*{Contexto y problemática}
    \noindent En la constante evolución de la movilidad, los vehículos autónomos representan una innovación trascendental.
    No obstante, el desafío primordial reside en dotar a estos vehículos con la capacidad de identificar y reaccionar ante situaciones de riesgo
    de manera precisa y oportuna. La detección temprana de posibles colisiones, amenazas viales y transgresiones graves a las normativas de tráfico
    es un aspecto esencial para garantizar la seguridad y la eficacia de estos sistemas autónomos. La visión computacional,
    utilizando las cámaras de video y sensores, se presenta como una estrategia central para esta detección,
    pero aún persisten desafíos tecnológicos significativos en la identificación y procesamiento oportuno de dichos eventos.
    \section*{Objetivos}
    \noindent{Objetivo general:}
    \newline
    \noindent Implementar un sistema de detección y evasión de colisiones basado en visión computacional para vehículos autónomos,
    \newline
    \newline
    \noindent{Objetivos específicos:}
    \begin{itemize}
        \item Modelar un ambiente de simulación donde un vehiculo circule por calles transitadas.
        \item Obtener datos de los sensores del vehiculo en simulación.
        \item Interpretar los datos de los sensores mediante técnicas de visión computacional.
        \item Procesar los datos y aprender a reaccionar.
    \end{itemize}
    \clearpage
    
    \section*{Estado del arte - Trabajos previos relacionados}
    \newline
    \begin{longtable}
        \hline
        \noindent \textbf{Referencia:}~\cite{bachute2021autonomous}                                    \\
        \textbf{Título:}
        Autonomous Driving Architectures: Insights of Machine Learning and Deep Learning Algorithms \\
        \textbf{Autores:}
        Bachute, Mrinal R and Subhedar, Javed M                                                        \\
        \textbf{Año:}
        2021                                                                                           \\
        \textbf{Revista:}
        Machine Learning with Applications                                                             \\
        \textbf{URL:}
        \url{https://www.sciencedirect.com/science/article/pii/S2666827021000827}                      \\
        \textbf{Resumen:}
        El artículo ofrece una visión general de cómo se aplican algoritmos de Aprendizaje Automático y Aprendizaje
        Profundo en sistemas de conducción autónoma y se enfoca en la evaluación de su desempeño en diversas tareas clave.
        Se analizan diversas tareas en la conducción autónoma, incluyendo la planificación de movimiento, la localización del vehículo,
        la detección de peatones, la detección de señales de tráfico, la detección de marcas viales, el estacionamiento automatizado,
        la ciberseguridad del vehículo y el diagnóstico de fallas del sistema.
        
        \newline
        \hline
        \noindent \textbf{Referencia:}~\cite{cai2021vision}                                            \\
        \textbf{Título:}
        Vision-based autonomous car racing using deep imitative reinforcement learning                 \\
        \textbf{Autores:}
        Cai, Peide and Wang, Hengli and Huang, Huaiyang and Liu, Yuxuan and Liu                        \\
        \textbf{Año:}
        2021                                                                                           \\
        \textbf{Revista:}
        IEEE Robotics and Automation Letters                                                           \\
        \textbf{URL:}
        \url{https://ieeexplore.ieee.org/abstract/document/9488179}                                    \\
        \textbf{Resumen:}
         se presenta un enfoque general de aprendizaje profundo imitativo y de refuerzo (DIRL) que logra el automovilismo
         autónomo ágil utilizando entradas visuales.
         Se valida el algoritmo en una simulación de conducción de alta fidelidad y en un automóvil RC a escala 1/20
         en el mundo real con capacidad computacional limitada.
         Los resultados de la evaluación muestran que el método supera a los enfoques anteriores de IL y RL en eficiencia.
        
        \newline
        \hline
        \noindent \textbf{Referencia:}~\cite{althoff2009model}                                         \\
        \textbf{Título:}
        Model-based probabilistic collision detection in autonomous driving                            \\
        \textbf{Autores:}
        Althoff, Matthias and Stursberg, Olaf and Buss                                                 \\
        \textbf{Año:}
        2009                                                                                           \\
        \textbf{Revista:}
        IEEE Transactions on Intelligent Transportation Systems                                        \\
        \textbf{URL:}
        \url{https://ieeexplore.ieee.org/abstract/document/4895669}                                    \\
        \textbf{Resumen:}                                                                              \\
        El artículo se centra en la seguridad de los caminos planificados para autos autónomos en relación con otros participantes en el tráfico.
        Se predice la ocupación de la carretera por otros vehículos de manera estocástica.
        La predicción tiene en cuenta las incertidumbres derivadas de las mediciones y los posibles comportamientos de los otros participantes en el tráfico.
        
        \newline
        \hline
        \noindent \textbf{Referencia:}~\cite{pavel2022vision}                                          \\
        \textbf{Título:}
        Vision-based autonomous vehicle systems based on deep learning: A systematic literature review \\
        \textbf{Autores:}
        Pavel, Monirul Islam and Tan, Siok Yee and Abdullah, Azizi                                     \\
        \textbf{Año:}
        2022                                                                                           \\
        \textbf{Revista:}
        Applied Sciences                                                                               \\
        \textbf{URL:}
        \url{https://www.mdpi.com/2076-3417/12/14/6831}                                                \\
        \textbf{Resumen:}                                                                              \\
        El artículo realiza una revisión sistemática de la literatura sobre el uso del aprendizaje profundo en AVS durante la última década.
        
        \clearpage
        
        \newline
        \hline
        \noindent \textbf{Referencia:}~\cite{alam2022cost}                                             \\
        \textbf{Título:}
        A cost-effective computer vision-based vehicle detection system                                \\
        \textbf{Autores:}
        Alam, Altaf and Jaffery, Zainul Abdin and Sharma, Himanshu                                     \\
        \textbf{Año:}
        2022                                                                                           \\
        \textbf{Revista:}
        Concurrent Engineering                                                                         \\
        \textbf{URL:}
        \url{https://journals.sagepub.com/doi/abs/10.1177/1063293X211069193}                           \\
        \textbf{Resumen:}                                                                              \\
        Propone un sistema de detección de vehículos basado en visión por computadora que utiliza un algoritmo
        de Gentle Adaptive Boosting con características tipo Haar para generar hipótesis de vehículos de manera rápida.
        
    \end{longtable}
    \clearpage
    
    
    \begin{table}
        \centering
        \caption{Tabla comparativa}
        \begin{tabular}{|p{4cm}|p{2cm}|p{2cm}|p{2cm}|p{2cm}|p{2cm}|}
            \hline
            \textbf{Características}
            & \textbf{Autonomous Driving Architectures \cite{bachute2021autonomous}}
            & \textbf{Vision-based Autonomous Car Racing \cite{cai2021vision}}
            & \textbf{Model-based Probabilistic Collision Detection \cite{althoff2009model}}
            & \textbf{Vision-based Autonomous Vehicle Systems \cite{pavel2022vision}}
            & \textbf{Cost-effective Vehicle Detection System \cite{alam2022cost}} \\
            \hline
            Uso de algoritmos de Aprendizaje Automático y Aprendizaje Profundo & X &   &   & X &   \\
            \hline
            Enfoque en la conducción autónoma                                  & X & X & X & X & X \\
            \hline
            Ventajas de la conducción autónoma                                 & X &   &   &   &   \\
            \hline
            Complejidad de los sistemas de conducción autónoma                 & X &   &   &   &   \\
            \hline
            Análisis de tareas en la conducción autónoma                       & X &   &   &   &   \\
            \hline
            Evaluación y comparación de algoritmos                             & X & X &   &   &   \\
            \hline
            Predicción estocástica de ocupación de la carretera                &   &   & X &   &   \\
            \hline
            Eficiencia en cálculos intensivos                                  &   & X & X &   &   \\
            \hline
            Utilización de cámaras RGB como sensores principales               &   & X &   & X &   \\
            \hline
            Detección de vehículos en conducción autónoma                      &   &   &   &   & X \\
            \hline
        \end{tabular}
    \end{table}
    \clearpage
    \section*{Metodología}
    \noindent La metodología propuesta se fundamenta en un enfoque iterativo que abarca diversas etapas para la implementación del sistema de detección
    y evasión de colisiones en vehículos autónomos. En primera instancia, se establecerá un entorno de simulación realista que refleje las condiciones de tráfico habituales.
    Posteriormente, se procederá a la adquisición y procesamiento de datos provenientes de los sensores de dicho entorno simulado. La fase siguiente implicará el diseño
    y la implementación de algoritmos de visión computacional para la detección temprana de eventos críticos en tiempo real. Estos algoritmos serán sometidos a un proceso
    de entrenamiento y ajuste utilizando técnicas de aprendizaje automático. Finalmente, se llevarán a cabo pruebas exhaustivas y evaluaciones para validar la efectividad
    y la precisión del sistema propuesto en situaciones simuladas de riesgo vial.
    \section*{Calendario de actividades}
    \begin{table}[htbp]
        \centering
        \caption{Calendario de Actividades}
        \begin{tabular}{|>{\raggedright\arraybackslash}p{4cm}|p{6cm}|p{3cm}|}
            \hline
            \textbf{Actividad}                          & \textbf{Descripción}                                        & \textbf{Duración} \\ \hline
            Investigación Preliminar                    & Revisión bibliográfica y análisis de entornos de simulación & 2 meses \\ \hline
            Diseño y Configuración del Entorno Simulado & Configuración del entorno y modelos de comportamiento & 1 mes \\ \hline
            Adquisición y Procesamiento de Datos        & Recopilación y procesamiento de datos de sensores & 3 meses \\ \hline
            Desarrollo y Entrenamiento de Algoritmos    & Implementación y entrenamiento de algoritmos de detección & 4 meses \\ \hline
            Evaluación y Ajuste del Sistema             & Pruebas exhaustivas y ajustes del sistema                   & 2 meses           \\ \hline
            Documentación y Análisis de Resultados      & Documentación y análisis crítico de resultados & 1 mes \\ \hline
            Redacción y Presentación de la Tesis        & Redacción y preparación para defensa oral                   & 1 mes             \\ \hline
        \end{tabular}
    \end{table}
    \clearpage
    \section*{Referencias bibliográficas}
    \bibliographystyle{acm}
    \bibliography{referecias}
\end{document}
