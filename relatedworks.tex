\documentclass[10pt,letterpaper,final]{article}
\usepackage[utf8]{inputenc}
\usepackage{amsmath}
\usepackage{amsfonts}
\usepackage{amssymb}
\usepackage{graphicx}
\usepackage{kpfonts}
\usepackage{tabularx}
\usepackage{hyperref}
\usepackage{natbib}
\begin{document}
    \section*{Bibliographic References}
    Rubén Martínez González
    \newline
    \begin{longtable}
        \hline
        \noindent \textbf{Referencia:}~\cite{alam2022cost} \\
        \textbf{Título:} \\
        Autonomous Driving Architectures: Insights of Machine Learning and Deep Learning Algorithms \\
        \textbf{URL:}
        \url{https://www.sciencedirect.com/science/article/pii/S2666827021000827}
        \textbf{Resumen:} \\
        El artículo se centra en la investigación en el campo de la conducción autónoma y destaca que esta área está ganando
        impulso debido a las ventajas inherentes de los sistemas de conducción autónoma. Las principales ideas y temas tratados en el artículo son:
        \begin{itemize}
            \item Ventajas de la conducción autónoma, como la reducción de la intervención humana y la disociación del conductor del vehículo.
            \item La complejidad de los sistemas de conducción autónoma, que involucra la integración de múltiples subsistemas.
            \item Diversas tareas en la conducción autónoma, incluyendo la planificación de movimiento, la localización del vehículo,
            la detección de peatones, la detección de señales de tráfico, la detección de marcas viales, el estacionamiento automatizado,
            la ciberseguridad del vehículo y el diagnóstico de fallas del sistema.
            \item Uso de algoritmos de Aprendizaje Automático y Aprendizaje Profundo en arquitecturas de conducción autónoma para realizar estas tareas.
            \item Evaluación y comparación de algoritmos basada en métricas como mIoU, AP, tasa de detección perdida, tasa de omisión,
            falsos positivos por imagen y promedio de detección de fotogramas falsos.
            \item Organización del estudio en función de las diferentes tareas del sistema de conducción autónoma.
        \end{itemize}
        el artículo ofrece una visión general de cómo se aplican algoritmos de Aprendizaje Automático y Aprendizaje
        Profundo en sistemas de conducción autónoma y se enfoca en la evaluación de su desempeño en diversas tareas clave. \\
        \pagebreak
        \hline
        \noindent \textbf{Referencia:}~\cite{althoff2009model} \\
        \textbf{Título:} \\
        Vision-based autonomous car racing using deep imitative reinforcement learning \\
        \textbf{URL:}
        \url{https://ieeexplore.ieee.org/abstract/document/9488179} \\
        \textbf{Resumen:}
        \begin{itemize}
            \item El automovilismo autónomo es un desafío en el campo del control robótico, que históricamente ha requerido mapas precisos,
            localización y planificación, lo que lo hace computacionalmente ineficiente y sensible a cambios en el entorno.
            \item  Recientemente, se han desarrollado sistemas de aprendizaje profundo de extremo a extremo que muestran resultados prometedores
            en la conducción/racing autónoma.Sin embargo, estos sistemas suelen basarse en aprendizaje por imitación supervisada (IL),
            que enfrenta problemas de discrepancia en la distribución de datos.
            \item También se han utilizado métodos de aprendizaje por refuerzo (RL), pero requieren una gran cantidad de datos de interacción riesgosa.
            \item se presenta un enfoque general de aprendizaje profundo imitativo y de refuerzo (DIRL) que logra el automovilismo
            autónomo ágil utilizando entradas visuales.
            \item El conocimiento de conducción se adquiere tanto del aprendizaje por imitación como del aprendizaje basado en modelos de RL,
            permitiendo al agente aprender de instructores humanos y mejorar su rendimiento interactuando con un modelo de mundo offline.
            \item Se valida el algoritmo en una simulación de conducción de alta fidelidad y en un automóvil RC a escala 1/20
            en el mundo real con capacidad computacional limitada.
            \item  Los resultados de la evaluación muestran que el método supera a los enfoques anteriores de IL y RL en eficiencia
            de muestra y rendimiento en la tarea.
        \end{itemize} \\
        \hline
        \pagebreak
        \hline
        \noindent \textbf{Referencia:}~\cite{bachute2021autonomous} \\
        \textbf{Título:} \\
        Model-based probabilistic collision detection in autonomous driving \\
        \textbf{Resumen:} \\
        & (Resumen en inglés para la referencia 3)      \\

        \hline
        & \textbf{Referencia:}~\cite{cai2021vision}     \\
        & \textbf{Resumen en inglés:}                   \\
        & (Resumen en inglés para la referencia 4)      \\
        &                                               \\
        \hline
        & \textbf{Referencia:}~\cite{konecny2022motion} \\
        & \textbf{Resumen en inglés:}                   \\
        & (Resumen en inglés para la referencia 5)      \\
        &                                               \\
        \hline
        & \textbf{Referencia:}~\cite{li2022human}       \\
        & \textbf{Resumen en inglés:}                   \\
        & (Resumen en inglés para la referencia 6)      \\
        &                                               \\
        \hline
        & \textbf{Referencia:}~\cite{pavel2022vision}   \\
        & \textbf{Resumen en inglés:}                   \\
        & (Resumen en inglés para la referencia 7)      \\
        &                                               \\
        \hline
        & \textbf{Referencia:}~\cite{prasad2023design}  \\
        & \textbf{Resumen en inglés:}                   \\
        & (Resumen en inglés para la referencia 8)      \\
        &                                               \\
        \hline
        & \textbf{Referencia:}~\cite{sushma2023dynamic} \\
        & \textbf{Resumen en inglés:}                   \\
        & (Resumen en inglés para la referencia 9)      \\
        &                                               \\
        \hline
    \end{longtable}
%    \begin{itemize}
%        \item \cite{alam2022cost}
%        \item \cite{althoff2009model}
%        \item \cite{bachute2021autonomous}
%        \item \cite{cai2021vision}
%        \item \cite{konecny2022motion}
%        \item \cite{li2022human}
%        \item \cite{pavel2022vision}
%        \item \cite{prasad2023design}
%        \item \cite{sushma2023dynamic}
%    \end{itemize}
    \nocite{*}
    \bibliographystyle{acm}
    \bibliography{referecias}
\end{document}
