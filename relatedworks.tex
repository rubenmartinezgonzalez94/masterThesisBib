\documentclass[10pt,letterpaper,final]{article}
\usepackage[utf8]{inputenc}
\usepackage{amsmath}
\usepackage{amsfonts}
\usepackage{amssymb}
\usepackage{graphicx}
\usepackage{kpfonts}
\usepackage{tabularx}
\begin{document}
    Bibliographic References
    Rubén Martínez González
    
    \begin{itemize}
        \item \cite{alam2022cost}
        \item \cite{althoff2009model}
        \item \cite{bachute2021autonomous}
        \item \cite{cai2021vision}
        \item \cite{konecny2022motion}
        \item \cite{li2022human}
        \item \cite{pavel2022vision}
        \item \cite{prasad2023design}
        \item \cite{sushma2023dynamic}
    \end{itemize}
    
    \section*{Bibliographic References}
    \begin{center}
        \begin{tabularx}{\textwidth}{|X|}
            \hline
            & \textbf{Título:} \\
            & Autonomous Driving Architectures: Insights of Machine Learning and Deep Learning Algorithms~\cite{alam2022cost} \\
            & \textbf{Resumen en inglés:} \\
            & Research in Autonomous Driving is taking momentum due to the inherent advantages of autonomous driving
            systems. The main advantage being the disassociation of the driver from the vehicle reducing the human
            intervention. However, the Autonomous Driving System involves many subsystems which need to be integrated as
            a whole system. Some of the tasks include Motion Planning, Vehicle Localization, Pedestrian Detection,
            Traffic Sign Detection, Road-marking Detection, Automated Parking, Vehicle Cybersecurity, and System Fault
            Diagnosis. This paper aims to the overview of various Machine Learning and Deep Learning Algorithms used in
            Autonomous Driving Architectures for different tasks like Motion Planning, Vehicle Localization, Pedestrian
            Detection, Traffic Sign Detection, Road-marking Detection, Automated Parking, Vehicle Cybersecurity and
            Fault Diagnosis. This paper surveys the technical aspects of Machine Learning and Deep Learning Algorithms
            used for Autonomous Driving Systems. Comparison of these algorithms is done based on the metrics like mean
            Intersect in over Union (mIoU), Average Precision (AP)missed detection rate, miss rate False Positives Per
            Image (FPPI), and average number for false frame detection. This study contributes to picture a review of
            the Machine Learning and Deep Learning Algorithms used for Autonomous Driving Systems and is organized based
            on the different tasks of the system. \\
            & \textbf{Resumen en español:} \\
            & El artículo se centra en la investigación en el campo de la conducción autónoma y destaca que esta área
            está ganando impulso debido a las ventajas inherentes de los sistemas de conducción autónoma.
            Las principales ideas y temas tratados en el artículo son:
            \begin{itemize}
                \item Ventajas de la conducción autónoma, como la reducción de la intervención humana y la disociación
                del conductor del vehículo.
                \item La complejidad de los sistemas de conducción autónoma, que involucra la integración de múltiples
                subsistemas.
                \item Diversas tareas en la conducción autónoma, incluyendo la planificación de movimiento, la
                localización del vehículo, la detección de peatones, la detección de señales de tráfico, la
                detección de marcas viales, el estacionamiento automatizado, la ciberseguridad del vehículo y el
                diagnóstico de fallas del sistema.
                \item Uso de algoritmos de Aprendizaje Automático y Aprendizaje Profundo en arquitecturas de conducción
                autónoma para realizar estas tareas.
                \item Evaluación y comparación de algoritmos basada en métricas como mIoU, AP, tasa de detección perdida
                , tasa de omisión, falsos positivos por imagen y promedio de detección de fotogramas falsos.
                \item Organización del estudio en función de las diferentes tareas del sistema de conducción autónoma.
            \end{itemize}
            el artículo ofrece una visión general de cómo se aplican algoritmos de Aprendizaje Automático y Aprendizaje
            Profundo en sistemas de conducción autónoma y se enfoca en la evaluación de su desempeño en diversas tareas clave.
            \\
            \hline
            & \textbf{Título:} \\
            & Vision-based autonomous car racing using deep imitative reinforcement learning~\cite{althoff2009model} \\
            & \textbf{Resumen en inglés:}
            %\cite{althoff2009model}
            & \textbf{Resumen en inglés:} \\
            & (Resumen en inglés para la referencia 2) \\
            & \\
            \hline
            %\cite{bachute2021autonomous}
            & \textbf{Resumen en inglés:} \\
            & (Resumen en inglés para la referencia 3) \\
            & \\
            \hline
            %\cite{cai2021vision}
            & \textbf{Resumen en inglés:} \\
            & (Resumen en inglés para la referencia 4) \\
            & \\
            \hline
            %\cite{konecny2022motion}
            & \textbf{Resumen en inglés:} \\
            & (Resumen en inglés para la referencia 5) \\
            & \\
            \hline
            %\cite{li2022human}
            & \textbf{Resumen en inglés:} \\
            & (Resumen en inglés para la referencia 6) \\
            & \\
            \hline
            %\cite{pavel2022vision}
            & \textbf{Resumen en inglés:} \\
            & (Resumen en inglés para la referencia 7) \\
            & \\
            \hline
            %\cite{prasad2023design}
            & \textbf{Resumen en inglés:} \\
            & (Resumen en inglés para la referencia 8) \\
            & \\
            \hline
            %\cite{sushma2023dynamic}
            & \textbf{Resumen en inglés:} \\
            & (Resumen en inglés para la referencia 9) \\
            & \\
            \hline
        \end{tabularx}
    \end{center}~\nocite{*}
    \bibliographystyle{acm}
    \bibliography{referecias}
\end{document}
