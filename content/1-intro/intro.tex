
El estacionamiento es una actividad esencial en la vida urbana, ya que permite a los conductores dejar sus vehículos en reposo mientras realizan otras actividades.
No obstante, el estacionamiento en áreas urbanas congestionadas presenta múltiples desafíos, incluyendo la falta de espacio,
la presencia de obstáculos y la visibilidad reducida, lo cual incrementa el riesgo de colisiones y daños a los vehículos.
Incluso habiendo identificado el espacio de estacionamiento, el proceso de maniobrar el vehículo para estacionar puede ser complicado y estresante,
especialmente para conductores con poca experiencia o en vehículos grandes.\\

Para mitigar estos problemas, se han desarrollado sistemas avanzados de asistencia al conductor, como el estacionamiento automático,
que facilitan esta tarea y mejoran la seguridad.
Sin embargo, la efectividad de estos sistemas depende en gran medida de la capacidad para estimar con precisión la posición del vehículo
con respecto al espacio de estacionamiento.
Una estimación incorrecta puede resultar en maniobras inseguras, especialmente en entornos con espacio limitado.\\

Representar esta ubicación de manera adecuada es crucial para el desarrollo de un sistema de estacionamiento automático confiable.
Para abordar esta cuestión, se propone diseñar una simulación en donde se obtendrán datos de sensores
y cámaras del vehículo para estimar su posición relativa al espacio de estacionamiento.\\

La simulación se llevará a cabo en un entorno controlado que refleja las condiciones reales de estacionamiento.
%Se modelará un escenario de estacionamiento con un vehículo y su espacio de estacionamiento, donde se adquirirán datos de sensores.
Dicho entorno consistirá de un cajón de estacionamiento objetivo, el vehículo que se va a controlar y objetos estáticos que pueden generar oclusiones.
El entorno de simulación permitirá extraer información a través de sensores los cuales permitirán hacer mediciones de las distancias y ángulos necesarios para maniobrar durante el estacionamiento.\\

La información geométrica extraída mediante técnicas de visión por computadora permitirá conocer la posición y orientación del vehículo respecto a la estructura del estacionamiento, proporcionando mediciones métricas reales del entorno tridimensional.
Esta información constituye un componente fundamental que puede ser utilizado por diversos sistemas de control y planificación de trayectorias en aplicaciones de conducción autónoma.\\

La investigación se centra en desarrollar un sistema robusto de visión por computadora que, mediante el procesamiento de imágenes de cámara y la reconstrucción geométrica del entorno de estacionamiento, proporcione información precisa de pose vehicular aplicable a sistemas de asistencia al conductor y estacionamiento automático en entornos urbanos.