
El estacionamiento es una actividad esencial en la vida urbana, ya que permite a los conductores dejar sus vehículos en reposo mientras realizan otras actividades.
No obstante, el estacionamiento en áreas urbanas congestionadas presenta múltiples desafíos, incluyendo la falta de espacio,
la presencia de obstáculos y la visibilidad reducida, lo cual incrementa el riesgo de colisiones y daños a los vehículos.
Incluso habiendo identificado el espacio de estacionamiento, el proceso de maniobrar el vehículo para estacionar puede ser complicado y estresante,
especialmente para conductores con poca experiencia o en vehículos grandes.\\

Para mitigar estos problemas, se han desarrollado sistemas avanzados de asistencia al conductor, como el estacionamiento automático,
que facilitan esta tarea y mejoran la seguridad.
Sin embargo, la efectividad de estos sistemas depende en gran medida de la capacidad para estimar con precisión la posición del vehículo
con respecto al espacio de estacionamiento.
Una estimación incorrecta puede resultar en maniobras inseguras, especialmente en entornos con espacio limitado.\\

Representar esta ubicación de manera adecuada es crucial para el desarrollo de un sistema de estacionamiento automático confiable.
Para abordar esta cuestión, se propone diseñar una simulación en donde se obtendrán datos de sensores
y cámaras del vehículo para estimar su posición relativa al espacio de estacionamiento.\\

La simulación se llevará a cabo en un entorno controlado que refleja las condiciones reales de estacionamiento.
%Se modelará un escenario de estacionamiento con un vehículo y su espacio de estacionamiento, donde se adquirirán datos de sensores.
Dicho entorno consistirá de un cajón de estacionamiento objetivo, el vehículo que se va a controlar y objetos estáticos que pueden generar oclusiones.
El entorno de simulación permitirá extraer información a través de sensores los cuales permitirán hacer mediciones de las distancias y ángulos necesarios para maniobrar durante el estacionamiento.\\

Las mediciones obtenidas, representadas mediante coordenadas cilíndricas, proporcionarán la información necesaria para entrenar algoritmos de aprendizaje automático.
Estos algoritmos podrán aprender y adaptarse a diversas condiciones y escenarios, mejorando la precisión y seguridad del sistema de estacionamiento automático.\\

La investigación se centra en desarrollar un sistema que, mediante la estimación de la posición del vehículo y la simulación de este entorno, permita avanzar en la creación de sistemas de estacionamiento automático eficientes y seguros en entornos urbanos congestionados.