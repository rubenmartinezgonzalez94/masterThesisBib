\noindent
RANSAC (Random Sample Consensus) es un paradigma de estimación robusta de modelos que alterna 
muestreos aleatorios mínimos con una evaluación de consenso (inliers) para encontrar hipótesis 
consistentes aun en presencia de una proporción elevada de datos atípicos. 
En cada iteración se selecciona un subconjunto mínimo de observaciones, se ajusta un modelo candidato 
y se mide cuántas observaciones concuerdan con él dentro de una tolerancia; tras múltiples iteraciones, 
se elige la hipótesis con mayor soporte y, opcionalmente, se refina con los inliers. 
Esta estrategia, ampliamente utilizada en visión por computadora y cartografía automatizada, 
fue introducida por Fischler y Bolles \cite{fischler1981ransac}.

\noindent
Beneficios clave del enfoque RANSAC:
\begin{itemize}
	\item Tolerancia a outliers y ruido: puede recuperar el modelo correcto aunque una parte sustancial de las observaciones sea espuria u ocluida.
	\item Flexibilidad de modelo: se aplica a múltiples familias (rectas, homografías, transformaciones, etc.).
	\item Sencillez operativa: alterna muestreo mínimo, ajuste y conteo de consenso.
\end{itemize}

\noindent
Variación propuesta en este trabajo. 
Como se explicará con más detalle en la Sección \ref{sec:metodo-ransac}, 
en esta investigación adaptamos RANSAC para estimar una homografía que alinee una retícula ideal 
con las líneas de los cajones de estacionamiento observadas en la imagen. 
Bajo las siguientes consideraciones generales: 
(i) se emplean dos puntos de fuga para agrupar las líneas en dos conjuntos según su punto de fuga; 
(ii) la muestra mínima se construye eligiendo dos líneas de cada conjunto (uno por cada punto de fuga) para formar cuatro intersecciones 
que hipotetizan un cajón; y 
(iii) a partir de esas esquinas se estima una homografía y se extiende una retícula \(n\times n\) 
sobre la imagen. Este planteamiento proporciona coherencia geométrica global del patrón 
y permite extrapolar la retícula más allá del campo visible, infiriendo la ubicación de cajones 
en puntos ciegos o parcialmente ocultos, lo cual resulta útil para la planificación 
y el estacionamiento automático.
