

RANSAC (Random Sample Consensus) es un algoritmo robusto para ajustar modelos a datos que pueden contener una proporción significativa de valores atípicos (outliers) \cite{fischler1981ransac,hartley2003multiple}. El objetivo es encontrar el modelo que mejor se ajusta a la mayor cantidad de datos consistentes (inliers), ignorando los outliers.

El algoritmo puede resumirse en los siguientes pasos \cite[Alg. 4.4]{hartley2003multiple}:
\begin{enumerate}
	\item Seleccionar aleatoriamente un subconjunto mínimo de $s$ datos de $S$ y ajustar el modelo a este subconjunto.
	\item Determinar el conjunto de datos $S_i$ que están a una distancia menor que un umbral $t$ del modelo. Este conjunto $S_i$ es el conjunto de consenso (inliers).
	\item Si el tamaño de $S_i$ es mayor que un umbral $T$, reestimar el modelo usando todos los puntos de $S_i$ y terminar.
	\item Si el tamaño de $S_i$ es menor que $T$, seleccionar un nuevo subconjunto y repetir los pasos anteriores.
	\item Tras $N$ iteraciones, seleccionar el mayor conjunto de consenso $S_i$ encontrado y reestimar el modelo usando todos los puntos de ese conjunto.
\end{enumerate}

Los parámetros principales del algoritmo son: el tamaño mínimo de muestra $s$, el umbral de distancia $t$, el umbral de consenso $T$ y el número máximo de iteraciones $N$. RANSAC es ampliamente utilizado en visión por computadora para estimar modelos como rectas, homografías y transformaciones, incluso en presencia de ruido y datos espurios.


\subsection{Variación propuesta en este trabajo}\label{sec:aply-ransac}
Como se explicará con más detalle en la sección \ref{sec:metodo-ransac}, 
en esta investigación adaptamos RANSAC para estimar una homografía que alinee una retícula ideal 
con las líneas de los cajones de estacionamiento observadas en la imagen. 
Bajo las siguientes consideraciones generales: 
\begin{itemize}
\item Se emplean dos puntos de fuga para agrupar las líneas en dos conjuntos según su punto de fuga; 
\item La muestra mínima se construye eligiendo dos líneas de cada conjunto (uno por cada punto de fuga) para formar cuatro intersecciones 
que hipotetizan un cajón; y 
\item A partir de esas esquinas se estima una homografía y se extiende una retícula \(n\times n\) 
sobre la imagen.
\end{itemize}

Este planteamiento proporciona coherencia geométrica global del patrón 
y permite extrapolar la retícula más allá del campo visible, infiriendo la ubicación de cajones 
en puntos ciegos o parcialmente ocultos, lo cual resulta útil para la planificación 
y el estacionamiento automático.
