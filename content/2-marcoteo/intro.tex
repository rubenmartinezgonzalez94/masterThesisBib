\noindent
Este capítulo reúne los fundamentos necesarios para comprender la metodología propuesta. Inicia con la formación de imagen y el modelo pinhole (Sección \ref{sec:modelo-pinhole}); continúa con la calibración y las homografías (Sección \ref{sec:calibracion-homografia}); y revisa las técnicas de detección de bordes y líneas empleadas, en particular Canny y la transformada de Hough (Sección \ref{sec:canny-hough}). Se establecen también las convenciones de marcos de referencia y la representación de la pose (Sección \ref{sec:pose-frames}), y se describen las plataformas y librerías utilizadas, como CARLA y OpenCV (Sección \ref{sec:plataformas}). Sobre esta base, se formaliza la geometría de las retículas de estacionamiento y sus supuestos (Sección \ref{sec:reticula-teoria}). Finalmente, se presenta la contribución central que guía la metodología: un esquema RANSAC adaptado para estimar una homografía que alinee una retícula ideal con las líneas observadas en la imagen, evaluando su consistencia geométrica mediante una función de costo robusta (Sección \ref{sec:ransac-teorico}).
