Este capítulo reúne los fundamentos teóricos y computacionales necesarios para comprender la metodología propuesta. Se inicia con la formación de imagen y el modelo de cámara pin-hole, explicando de manera intuitiva la proyección de puntos y líneas, así como la representación homogénea para operaciones geométricas (Sección \ref{subsec:camera}).

A continuación, se abordan las herramientas matemáticas para describir y manipular líneas y puntos en el plano imagen, incluyendo la obtención de líneas a partir de puntos, intersección de líneas, y el ajuste óptimo mediante formulaciones basadas en el espacio nulo, siguiendo la teoría de Kanatani y Hartley \& Zisserman.

Se presentan las homografías para el plano del suelo (Sección \ref{subsec:homografias}), y las técnicas de detección de bordes y líneas empleadas, en particular el método de Canny y la transformada de Hough (Sección \ref{sec:canny-hough}).

La sección central describe el algoritmo RANSAC para el ajuste robusto de modelos en presencia de outliers, detallando su funcionamiento y parámetros clave según la referencia clásica de Fischler y Bolles y la exposición de Hartley \& Zisserman (Sección \ref{sec:ransac-teorico}).

Finalmente, se establecen las convenciones de marcos de referencia y representación de la pose, y se describen las plataformas y bibliotecas utilizadas, como CARLA y OpenCV (Sección \ref{sec:plataformas}).
