La reconstrucción de coordenadas tridimensionales a partir de imágenes 2D capturadas por una cámara 
requiere comprender el modelo de proyección perspectiva y su inversión. En esta sección se presentan 
los fundamentos matemáticos necesarios para transformar puntos detectados en el plano imagen a coordenadas 
reales en el espacio 3D, específicamente para el caso de estructuras planas como la retícula de estacionamiento.

\subsection{Proyección perspectiva}

El modelo de cámara \emph{pin-hole} es la base de la geometría proyectiva en visión por computadora.
En este modelo, todos los rayos de luz que atraviesan el centro óptico (foco) de la cámara se proyectan 
sobre el plano imagen, generando una proyección perspectiva que preserva líneas rectas pero introduce 
distorsión en distancias y ángulos.

La transformación de un punto 3D en coordenadas de mundo \(\mathbf{P} = (X, Y, Z)^T\) a un punto 2D 
en el plano imagen \(\mathbf{p} = (u, v)^T\) está dada por la ecuación de proyección perspectiva:

\begin{equation}
s \begin{bmatrix} u \\ v \\ 1 \end{bmatrix} = 
\mathbf{K} \begin{bmatrix} \mathbf{R} & \mathbf{t} \end{bmatrix}
\begin{bmatrix} X \\ Y \\ Z \\ 1 \end{bmatrix}
\end{equation}

donde \(\mathbf{K}\) es la matriz de parámetros intrínsecos de la cámara, \(\mathbf{R}\) es la matriz 
de rotación, \(\mathbf{t}\) es el vector de traslación, y \(s\) es un factor de escala.

La matriz de parámetros intrínsecos tiene la forma:

\begin{equation}
\mathbf{K} = \begin{bmatrix}
f & 0 & c_x \\
0 & f & c_y \\
0 & 0 & 1
\end{bmatrix}
\end{equation}

donde \(f\) es la distancia focal en píxeles y \((c_x, c_y)\) es el centro óptico (punto principal) de la imagen.

\subsection{Proyección perspectiva de planos: la transformación T}

Para el caso particular de puntos que yacen en un plano (como el suelo del estacionamiento), 
la proyección 3D→2D se simplifica considerablemente. Asumiendo que la cámara está situada a una 
altura \(h\) sobre el plano del suelo y orientada hacia abajo, podemos definir un sistema de 
coordenadas donde el plano del suelo corresponde a \(Y = 0\).

La Figura~\ref{fig:proyeccion-3d-esquema} ilustra la geometría de esta proyección, donde:
\begin{itemize}
    \item El plano XY (plano horizontal) representa el suelo real con los espacios de estacionamiento
    \item El plano XZ (plano vertical) representa el plano imagen de la cámara
    \item El punto ``Foco'' representa el centro óptico de la cámara
    \item Los rayos amarillos muestran la proyección de puntos 3D al plano imagen
    \item Los puntos \(a, b, c, d, \ldots\) son coordenadas reales en el plano del suelo
    \item Los puntos \(T(a), T(b), T(c), T(d), \ldots\) son sus proyecciones en la imagen
\end{itemize}

\begin{figure}[!ht]
    \centering
    \includegraphics[width=0.99\textwidth]{img/2-mt/proyeccion_estereografica_foco_camara_suelo}
    \caption{Geometría de la proyección perspectiva de un plano 3D al plano imagen. Los cuadriláteros 
    con líneas verdes punteadas en el plano XY representan espacios de estacionamiento reales, mientras 
    que los cuadriláteros con líneas negras gruesas en el plano XZ representan su proyección en la imagen 
    capturada por la cámara ubicada en el Foco.}
    \label{fig:proyeccion-3d-esquema}
\end{figure}

La transformación que mapea puntos del plano del suelo (plano XY) al plano imagen (plano XZ) 
con distancias deformadas por la perspectiva está dada por:

\begin{equation}\label{eq:transform-T}
\begin{aligned}
T &: \mathbb{R}^2 \to \mathbb{R}^2 \\
T(x, y) &= \frac{1}{d + y} (dx, hy) = (u, v)
\end{aligned}
\end{equation}

donde:
\begin{itemize}
    \item \((x, y)\) son coordenadas en el plano del suelo (en metros)
    \item \((u, v)\) son coordenadas en el plano imagen (en píxeles o normalizadas)
    \item \(d\) es la distancia focal de la cámara
    \item \(h\) es la altura de la cámara sobre el nivel del suelo
    \item El término \(\frac{1}{d+y}\) introduce la distorsión perspectiva característica
\end{itemize}

Esta transformación \(T\) genera las \textit{distancias deformadas} observables en la imagen 2D, 
donde objetos más lejanos aparecen más pequeños y distancias iguales en el espacio 3D no se preservan 
en la proyección.

\subsection{Reconstrucción 3D: la transformación inversa \texorpdfstring{T\(^{-1}\)}{T⁻¹}}

Para realizar mediciones métricas en el espacio 3D a partir de puntos detectados en la imagen, 
es necesario invertir la proyección perspectiva. La transformación inversa \(T^{-1}\) permite 
recuperar coordenadas reales en el plano del suelo a partir de coordenadas en la imagen:

\begin{equation}\label{eq:transform-T-inv}
\begin{aligned}
T^{-1} &: \mathbb{R}^2 \to \mathbb{R}^2 \\
T^{-1}(u, v) &= \frac{1}{h - v} (hu, dv)
\end{aligned}
\end{equation}

Esta transformación ``desproyecta'' los puntos de la imagen al plano del suelo, deshaciendo 
la distorsión perspectiva. El factor \(\frac{1}{h-v}\) compensa la deformación introducida 
por la proyección, permitiendo recuperar distancias euclidianas reales.

\subsection{Reconstrucción mediante intersección de rayos}

En la práctica, la transformación \(T^{-1}\) se implementa mediante el método de intersección 
de rayos con el plano del suelo. El proceso consta de los siguientes pasos:

\begin{enumerate}
    \item \textbf{Normalización de coordenadas imagen:} Dado un punto \((u, v)\) en píxeles, 
    se aplica la matriz inversa de parámetros intrínsecos para obtener coordenadas normalizadas 
    (sin distorsión de la cámara):
    \begin{equation}
    \begin{bmatrix} x_n \\ y_n \\ 1 \end{bmatrix} = 
    \mathbf{K}^{-1} \begin{bmatrix} u \\ v \\ 1 \end{bmatrix}
    \end{equation}
    
    \item \textbf{Generación del rayo de proyección:} El vector \((x_n, y_n, 1)^T\) define 
    la dirección del rayo que pasa por el centro óptico de la cámara y el punto imagen.
    
    \item \textbf{Intersección con el plano del suelo:} Se escala el rayo hasta que su 
    componente Y alcanza el valor 0 (intersección con el plano \(Y=0\)):
    \begin{equation}
    \lambda = \frac{h}{y_n}
    \end{equation}
    
    \item \textbf{Coordenadas 3D en el plano del suelo:} El punto 3D resultante es:
    \begin{equation}
    \mathbf{P}_{suelo} = \lambda \begin{bmatrix} x_n \\ y_n \\ 1 \end{bmatrix} = 
    \begin{bmatrix} \lambda x_n \\ 0 \\ \lambda \end{bmatrix}
    \end{equation}
\end{enumerate}

Este método es equivalente a la transformación \(T^{-1}\) de la ecuación~\eqref{eq:transform-T-inv} 
y permite obtener coordenadas métricas reales que pueden utilizarse para calcular distancias, 
áreas y otras medidas geométricas en el espacio 3D del estacionamiento. Para obtener mediciones en metros en el espacio 3D, es necesario aplicar la proyección inversa \(T^{-1}\) utilizando los parámetros intrínsecos 
y extrínsecos de la cámara. La Sección~\ref{sec:metodo-extraccion-3d} describe la implementación 
práctica de este proceso.
