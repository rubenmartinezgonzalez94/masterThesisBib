El desarrollo y validación del sistema de visión por computadora permitió alcanzar los siguientes 
logros principales:

\subsubsection{Flujo de procesamiento completo de visión por computadora}

Se diseñó e implementó un sistema de extremo a extremo (\emph{end-to-end}) que procesa imágenes 
RGB de cámara y produce estimaciones de pose vehicular con coordenadas métricas en el espacio 3D. 
El flujo de procesamiento (\emph{pipeline}) integra múltiples técnicas clásicas de visión por 
computadora de manera coherente, demostrando que es posible construir sistemas de percepción 
efectivos sin depender exclusivamente de métodos de aprendizaje profundo.

\subsubsection{Adaptación de RANSAC para retículas regulares}

La variación propuesta de RANSAC aprovecha la estructura regular de los estacionamientos para 
mejorar la robustez y eficiencia del ajuste de homografías. A diferencia de aplicaciones estándar 
de RANSAC que trabajan con correspondencias de puntos, el método desarrollado opera directamente 
sobre líneas detectadas y aprovecha la información de puntos de fuga, resultando especialmente 
adecuado para escenarios con patrones geométricos repetitivos.

\subsubsection{Reconstrucción métrica 3D sin calibración externa}

El sistema calcula coordenadas tridimensionales métricas utilizando únicamente los parámetros 
intrínsecos de la cámara (derivados del campo de visión y resolución) y la altura conocida de 
la cámara sobre el suelo. No requiere calibración externa con patrones de referencia ni marcadores 
artificiales, lo cual facilita su implementación en vehículos reales.

\subsubsection{Validación mediante aplicación práctica}

En lugar de depender únicamente de métricas sintéticas, el sistema fue validado integrándolo en una aplicación de control real. 
Esta validación "de extremo a extremo" demuestra que la información proporcionada es 
consistente para guiar decisiones de control en tiempo real.

\subsection{Demostración de viabilidad}

\subsubsection{Integración con aprendizaje por refuerzo}

La integración del sistema de visión como componente de percepción en un entorno de RL demostró 
que es posible reemplazar observaciones perfectas por estimaciones visuales en aplicaciones de 
control autónomo. Aunque la tasa de éxito obtenida (18\%) es menor que la del entorno idealizado 
(93\%), este resultado debe interpretarse considerando la mayor complejidad del problema: el agente 
debe aprender políticas de control mientras lidia con incertidumbre perceptual realista.

\subsubsection{Transferibilidad a sistemas reales}

Al basarse en procesamiento de imágenes de cámara RGB estándar, el sistema es directamente 
transferible a vehículos reales equipados con cámaras. No depende de sensores especializados 
(LiDAR, GPS de alta precisión) ni de infraestructura externa (marcadores, balizas), lo cual 
facilita su implementación práctica en sistemas de asistencia al conductor comerciales.
