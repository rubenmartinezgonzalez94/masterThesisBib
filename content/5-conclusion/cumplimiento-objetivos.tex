Esta investigación tuvo como objetivo desarrollar un sistema de visión por computadora para la 
detección de retículas de estacionamiento y estimación de pose vehicular mediante procesamiento 
de imágenes. A continuación, se presenta el análisis del cumplimiento de cada uno de los objetivos 
específicos planteados en la Sección~\ref{sec:objetivos}:

\subsection{Implementación del entorno de simulación}

Se implementó exitosamente un entorno de simulación realista utilizando CARLA , 
configurando escenarios de estacionamiento con marcado de cajones, vehículos equipados con cámaras 
RGB frontales y física vehicular de alta fidelidad. El entorno permite capturar imágenes en 
resolución 1920×1080 a diferentes posiciones del vehículo, 
proporcionando datos sintéticos pero realistas para el desarrollo y prueba de algoritmos de 
visión por computadora.

\subsection{Desarrollo de algoritmos de procesamiento de imágenes}

Se diseñó e implementó un flujo de procesamiento (\emph{pipeline}) completo de imágenes que incluye:
\begin{itemize}
    \item Preprocesamiento mediante conversión a escala de grises
    \item Detección de bordes utilizando el algoritmo de Canny con umbrales adaptativos
    \item Extracción de líneas mediante la Transformada de Hough Probabilística
    \item Agrupamiento (\emph{clustering}) de líneas detectadas según su orientación para identificar dos familias 
    principales de líneas correspondientes a las direcciones de los cajones de estacionamiento
\end{itemize}

Este pipeline demuestra ser efectivo para detectar las líneas de marcado de cajones incluso 
en presencia de ruido y oclusiones parciales.

\subsection{Método basado en RANSAC para ajuste de retículas}

Se desarrolló una variación del algoritmo RANSAC adaptada específicamente para el problema de 
ajuste de retículas de estacionamiento. El método propuesto:
\begin{itemize}
    \item Estima puntos de fuga a partir de las familias de líneas detectadas
    \item Genera candidatos de cajones mediante muestreo aleatorio de líneas
    \item Calcula homografías que mapean cuadrados canónicos a cuadriláteros detectados
    \item Evalúa la consistencia entre retículas proyectadas y líneas reales
    \item Selecciona la mejor homografía mediante criterios de consenso
\end{itemize}

Esta adaptación de RANSAC resulta robusta frente a outliers y permite detectar la estructura 
regular de los estacionamientos de manera confiable.

\subsection{Reconstrucción 3D mediante proyección perspectiva inversa}

Se implementó el método de intersección de rayos con el plano del suelo para obtener coordenadas 
tridimensionales métricas a partir de puntos detectados en la imagen. El sistema:
\begin{itemize}
    \item Calcula la matriz de parámetros intrínsecos de la cámara a partir del campo de visión 
    y resolución de imagen
    \item Aplica la transformación inversa de la matriz intrínseca para generar rayos de proyección
    \item Escala los rayos hasta intersectar el plano Y=0 (nivel del suelo)
    \item Obtiene coordenadas (X, Z) métricas en el plano del estacionamiento
\end{itemize}

Este procedimiento permite calcular distancias euclidianas reales entre puntos de interés, 
proporcionando información geométrica precisa del entorno.

\subsection{Validación mediante integración en aplicación de control}

Se validó la utilidad del sistema integrándolo como componente de percepción en un entorno de 
aprendizaje por refuerzo para estacionamiento automático. Los resultados demostraron que:
\begin{itemize}
    \item El sistema de visión proporciona estimaciones de estado suficientemente informativas 
    para permitir el aprendizaje de políticas de control
    \item Un agente de RL logró una tasa de éxito del 18\% en la tarea de estacionamiento 
    utilizando la información visual procesada
    \item La mejora progresiva en las métricas de entrenamiento (recompensa, tasa de éxito) 
    evidencia que el sistema extrae características geométricas relevantes
\end{itemize}

Esta validación confirma que el sistema desarrollado puede ser utilizado en aplicaciones prácticas 
de conducción autónoma como módulo de percepción.

\subsection{Cumplimiento del objetivo general}

El objetivo general de desarrollar un sistema de visión por computadora para la detección de 
retículas de estacionamiento y estimación de pose vehicular se cumplió satisfactoriamente. El 
sistema resultante integra técnicas clásicas de visión por computadora (Canny, Hough, RANSAC) 
con geometría proyectiva 3D para proporcionar información métrica precisa del entorno de 
estacionamiento, aplicable a diversos sistemas de asistencia al conductor y aplicaciones de 
conducción autónoma.
