Este trabajo presenta un sistema completo de visión por computadora para la detección de retículas 
de estacionamiento y estimación de pose vehicular, demostrando que técnicas clásicas de procesamiento 
de imágenes pueden proporcionar información geométrica precisa y útil para aplicaciones de conducción 
autónoma.

El sistema desarrollado integra exitosamente algoritmos de detección de bordes (Canny), extracción 
de líneas (Hough), ajuste robusto de modelos geométricos (RANSAC) y reconstrucción tridimensional 
mediante proyección perspectiva inversa. Esta combinación de técnicas resulta en un flujo de 
procesamiento (\emph{pipeline}) robusto capaz de procesar imágenes de cámara en tiempo real y 
producir estimaciones de pose con coordenadas métricas en el espacio 3D.

La validación mediante integración en una aplicación de aprendizaje por refuerzo confirma que el 
sistema proporciona información suficientemente precisa y consistente para guiar decisiones de 
control en tareas complejas de navegación autónoma. Aunque los resultados de la aplicación de RL 
muestran margen de mejora, lo relevante es que el agente logró aprender políticas viables 
utilizando únicamente información visual procesada, validando la utilidad del sistema de visión 
como componente de percepción.

Las contribuciones principales del trabajo son:
\begin{itemize}
    \item Un método adaptado de RANSAC específico para ajuste de retículas regulares en entornos 
    de estacionamiento
    \item La clarificación conceptual entre homografías planares (útiles para detección 2D) y 
    proyección perspectiva 3D (necesaria para mediciones métricas)
    \item La demostración de viabilidad de sistemas de percepción basados en visión clásica como 
    alternativa a métodos exclusivamente basados en aprendizaje profundo
    \item Una implementación completa de extremo a extremo (\emph{end-to-end}) desde imágenes RGB hasta coordenadas 3D métricas
\end{itemize}

El trabajo sienta las bases para futuras investigaciones que podrían extender el sistema mediante 
métodos híbridos (combinando técnicas clásicas con aprendizaje profundo), optimizaciones de 
eficiencia computacional, y validación en vehículos reales. La arquitectura modular desarrollada 
facilita la experimentación con mejoras incrementales sin necesidad de rediseñar el sistema completo.

En conclusión, se ha demostrado que es posible desarrollar sistemas de visión por computadora 
efectivos para estimación de pose vehicular en entornos de estacionamiento utilizando únicamente 
cámaras RGB estándar y técnicas clásicas de procesamiento de imágenes, proporcionando una 
alternativa viable y comprensible a enfoques puramente basados en datos que requieren grandes 
cantidades de ejemplos de entrenamiento anotados.
