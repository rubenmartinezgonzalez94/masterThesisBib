\noindent
%    problemas en los estacionamientos
Los estacionamientos son imprescindibles en la vía urbana, ya que permiten a los conductores estacionar sus vehículos
de manera segura y eficiente. Sin embargo, el proceso de estacionamiento puede ser complicado y estresante,
especialmente en áreas congestionadas con espacio limitado y visibilidad reducida.
Factores como la falta de espacio, la presencia de obstáculos y la poca visibilidad para el conductor ocasionan
dificultades al estacionar un vehículo, lo que puede aumentar el riesgo de colisiones y daños al vehículo.
\\
%   se han logrado sistemas de asistencia al conductor
En la actualidad, la búsqueda de soluciones para mejorar la eficiencia y seguridad en el desplazamiento vehicular
ha llevado al desarrollo de sistemas avanzados de asistencia al conductor.
Entre estos sistemas, el estacionamiento automático ha ganado relevancia como una función que puede contribuir
a reducir los riesgos asociados con el estacionamiento en entornos urbanos congestionados.
\\
%   problemas de sistemas de estacionamiento automático
Sin embargo, el desarrollo de sistemas de estacionamiento automático presenta desafíos significativos,
especialmente en lo que respecta a la estimación de la posición del vehículo con respecto al espacio de estacionamiento.
El cálculo incorrecto de esta posición puede resultar en maniobras de estacionamiento inseguras o peligrosas,
especialmente en entornos donde el espacio de estacionamiento es limitado o con poca visibilidad para el conductor.
\\
%   necesidad de sistemas de asistencia al conductor más autónomos
En este contexto, continua la necesidad de desarrollar soluciones de utilidad para que los sistemas de asistencia al conductor
sean cada vez más autosuficientes y no dependan de la intervención limitada del conductor.
\\
%   en que consiste la investigación
Esta investigación se enfoca en poder estimar la posición de un vehículo con respecto a su espacio de estacionamiento
utilizando cámaras y sensores y utilizar esta posición estimada para lograr un sistema de estacionamiento automático en simulación.