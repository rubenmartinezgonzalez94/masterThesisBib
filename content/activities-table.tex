\begin{center}
    \begin{tabularx}{\textwidth}{|c|p{4cm}|p{1cm}|p{1cm}|X|}
        \hline
        \textbf{Nº} & \textbf{Actividad} & \textbf{Inicio} & \textbf{Entrega} & \textbf{Detalles} \\
        \hline
        1 & Revisión de trabajos relacionados con vehículos autónomos & Enero 2024 & Marzo 2024 & Revisión de literatura sobre vehículos autónomos, incluyendo técnicas de percepción y control. \\
        \hline
        2 & Revisión de trabajos relacionados con parqueo automático & Enero 2024 & Marzo 2024 & Revisión de literatura sobre sistemas de parqueo automático, enfocándose en algoritmos de detección y maniobra. \\
        \hline
        3 & Selección de tecnología para simulación & Enero 2024 & Abril 2024 & Selección de herramientas y tecnologías para la simulación, como CARLA y otros simuladores de vehículos. \\
        \hline
        4 & Revisión de técnicas de detección de contornos & Marzo 2024 & Julio 2024 & Estudio de métodos para detectar contornos en imágenes, incluyendo técnicas de procesamiento de imágenes. \\
        \hline
        5 & Revisión de técnicas de detección de líneas & Marzo 2024 & Julio 2024 & Estudio de métodos para detectar líneas en imágenes, como la transformada de Hough. \\
        \hline
        6 & Revisión de técnicas para manejo de homografías & Marzo 2024 & Julio 2024 & Estudio de técnicas para manejar homografías en imágenes, aplicadas a la reconstrucción de escenas. \\
        \hline
        7 & Instalación de CARLA Simulator & Abril 2024 & Mayo 2024 & Instalación del simulador CARLA, configurando el entorno de desarrollo y pruebas. \\
        \hline
        8 & Estudio de la documentación de CARLA Simulator & Abril 2024 & Octubre 2024 & Lectura y comprensión de la documentación de CARLA, incluyendo ejemplos y casos de uso. \\
        \hline
        9 & Diseño y Configuración del Entorno de simulación & Mayo 2024 & Octubre 2024 & Configuración del entorno de simulación en CARLA, incluyendo la creación de escenarios y la integración de sensores. \\
        \hline
        10 & Adquisición de datos de sensores en simulación & Mayo 2024 & Octubre 2024 & Obtención de datos de sensores en el entorno simulado, como cámaras. \\
        \hline
        11 & Extracción de imágenes RGB de maniobras de estacionamiento & Octubre 2024 & Octubre 2024 & Obtención de imágenes de maniobras de estacionamiento, capturando diferentes ángulos y condiciones. \\
        \hline
        12 & Extracción de contornos relevantes de las imágenes & Octubre 2024 & Octubre 2024 & Detección de contornos importantes en las imágenes, utilizando técnicas de procesamiento de bordes. \\
        \hline
        13 & Extracción de líneas de los contornos & Octubre 2024 & Octubre 2024 & Detección de líneas en los contornos de las imágenes, aplicando algoritmos de detección de líneas. \\
        \hline
        14 & Extracción de las ecuaciones de las líneas & Octubre 2024 & Diciembre 2024 & Cálculo de ecuaciones de las líneas detectadas, representando las líneas en un sistema de coordenadas. \\
        \hline
        15 & Extracción de intersecciones de las líneas & Octubre 2024 & Diciembre 2024 & Cálculo de intersecciones de las líneas detectadas, determinando puntos clave en la escena. \\
        \hline
        16 & Selección del primer punto de fuga mediante clustering & Diciembre 2024 & Enero 2025 & Identificación del primer punto de fuga usando clustering, agrupando puntos de intersección relevantes. \\
        \hline
        17 & Selección del segundo punto de fuga mediante lógica geométrica & Diciembre 2024 & Enero 2025 & Identificación del segundo punto de fuga usando lógica geométrica, analizando la disposición espacial. \\
        \hline
        18 & Reconstrucción de la retícula de estacionamiento & Diciembre 2024 & Febrero 2025 & Reconstrucción de la retícula de estacionamiento, utilizando los puntos de fuga y las líneas detectadas. \\
        \hline
        19 & Mejora de la reconstrucción de la retícula (Filtro de Kalman) & Diciembre 2024 & Febrero 2025 & Optimización de la retícula usando el filtro de Kalman, mejorando la precisión de la reconstrucción. \\
        \hline
        20 & Representación de homografía correspondiente en 3D & Enero 2025 & Marzo 2025 & Representación de la homografía en 3D, visualizando la relación espacial entre el vehículo y el estacionamiento. \\
        \hline
        21 & Seguimiento de uno de los cajones de estacionamiento & Enero 2025 & Marzo 2025 & Seguimiento de un cajón de estacionamiento específico, monitoreando su posición y orientación. \\
        \hline
        22 & Cálculo de las distancias a las cuatro esquinas del cajón & Enero 2025 & Marzo 2025 & Medición de distancias a las esquinas del cajón, utilizando coordenadas cilíndricas para la representación. \\
        \hline
        23 & Representación de pose del vehículo relativa al cajón & Enero 2025 & Marzo 2025 & Representación de la posición del vehículo respecto al cajón, facilitando la maniobra de estacionamiento. \\
        \hline
        24 & Diseño e implementación de un enviroment RL & Marzo 2025 & Mayo 2025 & Creación de un entorno de aprendizaje por refuerzo, configurando el entorno y las condiciones de entrenamiento. \\
        \hline
        25 & Diseño del action space y observation space & Marzo 2025 & Mayo 2025 & Definición del espacio de acciones y observaciones, especificando las posibles acciones y estados del agente. \\
        \hline
        26 & Diseño de función de recompensa & Marzo 2025 & Mayo 2025 & Creación de la función de recompensa para el agente RL, incentivando comportamientos deseados durante el entrenamiento. \\
        \hline
        27 & Entrenamiento de un agente RL para estacionamiento & Abril 2025 & Mayo 2025 & Entrenamiento del agente de RL para estacionamiento, utilizando el entorno y la función de recompensa definidos. \\
        \hline
        28 & Experimentación y documentación de resultados & Mayo 2025 & Junio 2025 & Realización de experimentos y documentación de resultados, evaluando el desempeño del agente RL. \\
        \hline
        29 & Redacción y Revisión del documento de tesis & Octubre 2024 & Junio 2025 & Escritura y revisión del documento de tesis, compilando los resultados y conclusiones del proyecto. \\
        \hline
    \end{tabularx}
\end{center}