
En esta sección implementamos el modelo de aprendizaje por refuerzo combinando Highway-env con CARLA,
utilizando los resultados de detección de retícula como sistema de observación. La implementación
mantiene la estructura probada de Highway-env pero integra elementos realistas a través de wrappers
personalizados que conectan con el simulador CARLA y nuestro sistema de visión por computadora.

\subsection{Configuración del modelo}

Utilizamos el algoritmo SAC (Soft Actor-Critic) de Stable-Baselines3 (sección~\ref{sec:stable-baselines3})
con Hindsight Experience Replay (HER) para manejar la naturaleza de objetivo disperso del estacionamiento.
La configuración sigue las mejores prácticas de Highway-env (sección~\ref{subsec:highway-env-theory}):

\begin{itemize}
    \item \textbf{Algoritmo}: SAC con política \texttt{MultiInputPolicy} para manejar espacios de observación estructurados
    \item \textbf{HER}: 4 objetivos muestreados por episodio con estrategia ``future'' para aprendizaje de objetivos dispersos
    \item \textbf{Hiperparámetros}: Buffer de 1M transiciones, learning rate 1e-3, red neuronal [256, 256, 256]
    \item \textbf{Entorno base}: \texttt{parking-v0} de Highway-env con modificaciones mediante wrappers
\end{itemize}

\subsection{Arquitectura de wrappers}

Implementamos tres wrappers de Gymnasium (sección~\ref{sec:gymnasium}) que adaptan el entorno original
manteniendo compatibilidad con la interfaz estándar:


\textbf{CarlaInitRoadWrapper}: Construye la geometría del estacionamiento usando coordenadas reales de CARLA
(sección~\ref{sec:carla-teorico}). Reemplaza el \texttt{RoadNetwork} predeterminado con cajones de estacionamiento
generados a partir de la posición objetivo actual en el simulador, creando un layout de 7 cajones centrado
en el objetivo.


\textbf{CarlaObservationWrapper}: Sustituye las observaciones idealizadas de Highway-env por estimaciones
realistas obtenidas mediante nuestro sistema de visión por computadora. Aplica los algoritmos de detección
de retícula desarrollados (sección~\ref{sec:metodo-reticula}) para extraer coordenadas $(x,y)$ y orientación
$\theta$ del vehículo, complementadas con velocidad del sensor vehicular de CARLA.


\textbf{CarlaActionWrapper}: Traduce las acciones continuas 
aceleración, dirección del agente de RL
a controles vehiculares en CARLA, aplicando throttle y steering al vehículo simulado 
y sincronizando
el tiempo de simulación.

\subsection{Integración del sistema de visión}

El componente crítico de nuestra adaptación reside en el \texttt{CarlaObservationWrapper}, que implementa
el pipeline completo de visión por computadora como intérprete de observaciones:

\begin{enumerate}
    \item \textbf{Captura de imagen}: Obtiene el frame de la cámara frontal del vehículo en CARLA
    \item \textbf{Detección de retícula}: Aplica los algoritmos desarrollados (Canny, Hough, RANSAC)
    para identificar la estructura del estacionamiento
    \item \textbf{Estimación de pose}: Convierte la retícula detectada en coordenadas del vehículo
    relativas al estacionamiento mediante el procedimiento descrito en la metodología
    \item \textbf{Construcción de observación}: Formatea las estimaciones en el vector de estado
    esperado por el agente de RL: $[x, y, v_x, v_y, \cos(\theta), \sin(\theta)]$
\end{enumerate}


Esta integración permite que el agente aprenda políticas de estacionamiento basándose en información
visual procesada en tiempo real, replicando las condiciones que enfrentaría un sistema autónomo real
donde la localización debe inferirse de sensores imperfectos en lugar de coordenadas globales ideales.

\subsection{Ventajas de la implementación híbrida}

La arquitectura propuesta combina las fortalezas de cada componente:
\begin{itemize}
    \item \textbf{Highway-env}: Funciones de recompensa validadas y convergencia probada de algoritmos
    \item \textbf{CARLA}: Física vehicular realista y renderizado visual de alta fidelidad
    \item \textbf{Sistema de visión}: Percepción basada en cámara con incertidumbre realista
    \item \textbf{Gymnasium/SB3}: Interfaz estándar y algoritmos de RL de última generación
\end{itemize}


Esta configuración permite evaluar tanto la robustez del sistema de detección de retícula
como su viabilidad en aplicaciones de control autónomo, demostrando que es posible sustituir
sensores idealizados por sistemas de percepción visual en tareas complejas de navegación.
