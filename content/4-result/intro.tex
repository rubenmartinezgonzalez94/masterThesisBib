Los capítulos anteriores presentaron el desarrollo de un sistema de visión por computadora 
para la detección y reconstrucción tridimensional de retículas de estacionamiento mediante 
técnicas de procesamiento de imágenes (Capítulo~\ref{chap:marco-teorico}) y su implementación 
práctica utilizando detección de bordes, transformada de Hough, ajuste mediante RANSAC y 
proyección perspectiva inversa (Capítulo~\ref{chap:metodologia}). Este sistema constituye 
la contribución central del presente trabajo: un método capaz de extraer información geométrica 
métrica del entorno de estacionamiento a partir de imágenes de cámara, proporcionando coordenadas 
3D reales de la estructura del estacionamiento y la posición relativa del vehículo.

El presente capítulo tiene como objetivo ilustrar aplicaciones prácticas de este sistema 
de visión por computadora en contextos de conducción autónoma y estacionamiento automático. 
En particular, se explora cómo el sistema desarrollado puede funcionar como un intérprete 
de estado que traduce información visual de cámaras a representaciones numéricas del estado del 
vehículo, las cuales pueden ser consumidas por sistemas de control o planificación de trayectorias.

Una aplicación natural de este tipo de sistema intérprete es su integración con algoritmos de 
aprendizaje por refuerzo (RL), donde el agente requiere observaciones del estado del sistema 
para aprender políticas de control. En este contexto, el sistema de visión por computadora actúa 
como el componente de percepción que proporciona al agente información sobre su posición, orientación 
y entorno, reemplazando sensores idealizados o sistemas de localización global que no estarían 
disponibles en escenarios reales.

Las secciones siguientes presentan una implementación demostrativa donde el sistema de detección 
de retículas desarrollado se integra como módulo de observación en un entorno de aprendizaje por 
refuerzo para estacionamiento automático. Esta aplicación permite evaluar la viabilidad 
de integración del sistema de visión en pipelines de control autónomo, demostrando cómo la 
información geométrica extraída puede ser utilizada por agentes de decisión en tiempo real.

Es importante destacar que el objetivo de este capítulo es ejemplificar casos de uso
del sistema de visión desarrollado, mostrando su utilidad como componente de percepción en 
arquitecturas de conducción autónoma. Los resultados presentados evalúan principalmente la 
capacidad del sistema para proporcionar información de estado útil en aplicaciones de control, 
más que el desempeño final del sistema de control en sí. El método de visión por computadora 
constituye una solución independiente y completa que puede integrarse en diversos contextos 
más allá del ejemplo específico de aprendizaje por refuerzo presentado aquí.
