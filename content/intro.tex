\noindent
El avance continuo en la tecnología de vehículos autónomos representa un logro significativo en la revolución del transporte.
Existe la necesidad de desarrollar sistemas ``inteligentes'' que permitan a estos vehículos aprender a conducir de manera autónoma y,
al mismo tiempo, detectar posibles colisiones y reaccionar de manera similar a como lo haría un conductor humano.
La convergencia de la inteligencia artificial, la visión computacional y los sistemas de control ha generado una nueva era en la movilidad,
desafiando y redefiniendo las fronteras de la conducción convencional.
Si bien los avances en la conducción autónoma han sido significativos, la detección y respuesta a
situaciones de peligro, como colisiones inminentes, siguen siendo un desafío complejo.\\ \newline
En este contexto, este trabajo se centra en el desarrollo de un sistema de detección y evasión de colisiones basado en visión computacional.
Para lograr este desarrollo se conformará un entorno simulado que refleje el escenario de conducción de un vehículo,
y mediante algoritmos de visión computacional se analizará la información proveniente de cámaras y sensores.
Con el procesamiento de estos datos se podrá modelar el medio que rodea al vehículo, lo que posibilitará
el desarrollo de estrategias de detección y evasión de colisiones en diferentes situaciones de manejo.
Una vez que se ha realizado el procesamiento exhaustivo de los datos provenientes de cámaras y sensores mediante técnicas de visión computacional,
se plantea la posibilidad de emplear algoritmos de aprendizaje por refuerzo.
Los algoritmos de aprendizaje por refuerzo, al recibir estos datos procesados como entrada, tienen la capacidad de aprender de manera progresiva
a tomar decisiones inteligentes en tiempo real.
Aprovechando la retroalimentación proporcionada por el entorno, estos algoritmos pueden mejorar continuamente su capacidad para reaccionar y evitar colisiones.\\ \newline
La seguridad en la carretera y la confianza del público en esta tecnología dependen en gran medida de la capacidad
de los vehículos autónomos para enfrentar situaciones de tráfico de manera eficiente y segura.
Esta investigación busca abordar esta problemática crítica, avanzando hacia el escenario en el que los vehículos autónomos
sean capaces de igualar e incluso superar las habilidades de conducción humana en términos de detección y respuesta
a situaciones de colisión.