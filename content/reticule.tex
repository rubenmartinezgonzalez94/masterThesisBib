\noindent
Cuando proyectamos un escena del mundo real en tres dimensiones a un plano de dos dimensiones como la película o el sensor de una cámara, se produce una transformación en la imagen.
Esta transformación, que se conoce como transformación proyectiva, provoca que las líneas paralelas en el mundo real al proyectarse en el plano de la cámara se intersecten en un punto. A dicho punto se le conoce como punto de fuga.

\begin{figure}[!ht]
    \centering
    \begin{subfigure}{0.4\textwidth}
        \includegraphics[width=\textwidth]{img/reticule/paralel_lines}\label {fig:parallel_lines}
        \caption{Ejemplo de líneas paralelas en un escenario real en 3 dimensiones}
    \end{subfigure}
    \begin{subfigure}{0.4\textwidth}
        \includegraphics[width=\textwidth]{img/reticule/pov}\label {fig:pov}
        \caption{Proyección de líneas paralelas en el plano de la cámara}
    \end{subfigure}

    \label{fig:distorion}
\end{figure}

\noindent
Dado que las líneas de los cajones de estacionamiento son paralelas por su geometría, forman patrones en una retícula de paralelogramos.
Esto permite utilizar técnicas de detección de líneas para identificar los puntos de fuga y estimar la posición de la retícula de estacionamiento.

\begin{figure}[!ht]
    \centering
    \begin{subfigure}{0.8\textwidth}
        \includegraphics[width=\textwidth]{img/reticule/pov_reticule}\label {fig:pov_reticule}
    \end{subfigure}
    \begin{subfigure}{0.8\textwidth}
        \includegraphics[width=\textwidth]{img/reticule/pov_reticule_layer}\label {fig:pov_reticule_layers}
    \end{subfigure}
    \caption{Líneas paralelas de la retícula de estacionamiento}
    \label{fig:reticule_pov}
\end{figure}

\noindent

\subsection{Detección de líneas paralelas:}

\begin{itemize}
    \item Área de interés:

    Teniendo en cuenta que la camara del vehículo se encuentra ubicada en la parte delantera a una altura conocida y en un ángulo paralelo al suelo, el área de interés de la imagen donde se encuentran las líneas de los cajones de estacionamiento quedara siempre por debajo del horizonte de la imagen.
    Por lo tanto, se puede eliminar la parte superior de la imagen para
    reducir el ruido y mejorar la detección de las líneas.
    \begin{figure}[!ht]
        \centering
        \includegraphics[width=0.6\textwidth]{img/reticule/horizont}
        \caption{Área de interés de la imagen}
        \label{fig:roi}
    \end{figure}

    \item Umbralización:

    Al área de interés de la imagen se le aplica una umbralización para realzar las líneas blancas de los cajones de estacionamiento y eliminar otros elementos no relevantes que puedan interferir en la detección.
    \begin{figure}[!ht]
        \centering
        \includegraphics[width=0.6\textwidth]{img/reticule/thresholded}
        \caption{Imagen umbralizada}
        \label{fig:threshold}
    \end{figure}

    \clearpage
    \item Detección de contornos:

    Se utiliza el algoritmo de Canny para detectar los bordes de las líneas en la imagen umbralizada.
    \begin{figure}[!ht]
        \centering
        \includegraphics[width=0.6  \textwidth]{img/reticule/canny}
        \caption{Detección de bordes mediante el algoritmo de Canny}
        \label{fig:edges}
    \end{figure}

    \item Detección de líneas:

    Se aplica la transformada de Hough para detectar las coordenadas de inicio y fin de las líneas en la imagen.
    \begin{figure}[!ht]
        \begin{subfigure}{0.5\textwidth}
            \includegraphics[width=\textwidth]{img/reticule/hough2}
            \caption{Líneas detectadas con la transformada de Hough}
            \label{fig:hough}
        \end{subfigure}
        \begin{subfigure}{0.5\textwidth}
            \includegraphics[width=\textwidth]{img/reticule/hough}
            \caption{Líneas detectadas en la imagen original}
            \label{fig:lines}
        \end{subfigure}
    \end{figure}

    \item Representación de las ecuaciones de las líneas:

    Una vez obtenidas las coordenadas de inicio y fin de cada línea paralela, podemos utilizar la ecuación general de la recta:
    \begin{equation}
        Ax + By + C = 0
    \end{equation}
    Esta ecuación nos permite determinar la orientación de cada línea.
    Dado que las coordenadas iniciales y finales de cada línea corresponden a los valores de $x$ y $y$, respectivamente,    estas se pueden emplear para formular un sistema de ecuaciones que describa los parámetros de la recta $[A, B, C]$.
    Dicho sistema puede representarse de manera matricial como sigue:
    \begin{equation}
        \begin{aligned}
            \left[\begin{array}{ccc}
                      x_1 & y_1 & 1 \\
                      x_2 & y_2 & 1
            \end{array}\right]
            \begin{bmatrix}
                A \\
                B \\
                C
            \end{bmatrix}
            =
            \begin{bmatrix}
                0 \\
                0
            \end{bmatrix}
        \end{aligned}
    \end{equation}

    Esta representación permite calcular de forma precisa los coeficientes de la ecuación de la recta para cada línea detectada, lo que es fundamental para analizar su orientación y posición dentro de la retícula de estacionamiento.

    \item Cálculo de ecuaciones de las líneas:

    Para calcular los coeficientes $[A, B, C]$ de las ecuaciones de las líneas detectadas, se utiliza el concepto de espacio nulo (\emph{null space}). Este enfoque se basa en el hecho de que cualquier vector en el espacio nulo de una matriz $\mathbf{M}$ satisface la ecuación
    \begin{equation}
        \mathbf{Mv} = 0
    \end{equation}

%Comentario: No has definido el vector v.


    Cada línea se representa mediante dos puntos $(x_1, y_1)$ y $(x_2, y_2)$. A partir de estas coordenadas homogeneas, construimos una matriz $\mathbf{M}$ de la forma:
    \[
        \mathbf{M} = \begin{bmatrix}
                         x_1 & y_1 & 1 \\
                         x_2 & y_2 & 1
        \end{bmatrix}
    \]
    Esta matriz define el sistema de ecuaciones que describe la recta que pasa por los puntos dados.
    El espacio nulo de $\mathbf{M}$ corresponde al conjunto de vectores $[A,B,C]$ que satisfacen
    \[
        \mathbf{M} \cdot \begin{bmatrix}
                             A \\ B \\ C
        \end{bmatrix} = \mathbf{0}.
    \]
    Se utiliza la Descomposición en Valores Singulares (SVD) para calcular este espacio nulo, ya que es una herramienta robusta y numéricamente estable.
    La SVD descompone la matriz \(M\) en tres matrices \(U\), \(S\) y \(V\), donde el espacio nulo de \(M\) se puede obtener a partir de la última columna de la matriz \(V\), que corresponde al vector singular más pequeño (el más cercano al cero).
    El vector resultante del espacio nulo se normaliza para que tenga una magnitud manejable.
    Esto asegura que los coeficientes \(A\), \(B\) y \(C\) sean comparables entre distintas líneas.

    \item Cálculo de intersecciones:

    Dado que ya se tendrían las ecuaciones de todas las líneas paralelas en el plano de la cámara, se pueden calcular las intersecciones de estas líneas realizando un producto cruz entre las ecuaciones homogéneas [\(A\),\(B\),\(C\)] de todos los pares de líneas.

    El resultado de este producto cruz es la coordenada homogénea de un punto en el espacio que corresponde a la intersección de las líneas.
    Si este punto es finito (cuando la tercera componente no es cero), se puede deshomogenizar para obtener las coordenadas cartesianas en el plano de la cámara.
    En cambio, si el punto es infinito (cuando la tercera componente muy cercana a cero), significa que las líneas son paralelas y se intersectan en el infinito.


    \item  Agrupación de intersecciones:
    %https://scikit-learn.org/stable/modules/clustering.html#hierarchical-clustering

    Analizando los puntos de intersecciones obtenidos que se encuentran en el plano de la cámara, se pueden agrupar para determinar donde están
    concentrada la mayor cantidad de intersecciones.
    Este punto de concentración de intersecciones debe corresponder al punto de fuga principal de la retícula de estacionamiento.
    \begin{figure}[!ht]
        \centering
        \includegraphics[width=0.8\textwidth]{img/reticule/svd-km}
        \caption{Agrupacion de intersecciones de las líneas detectadas}
        \label{fig:intersections}
    \end{figure}

    \\Para estimar la ubicación de este punto de fuga principal no es necesario tener en cuenta todas las intersecciones detectadas,
    sino solo aquellas que se encuentran en una zona cercana al horizonte de la imagen.
    Esto se debe a que, debido a la perspectiva de la cámara, las líneas paralelas en el mundo real convergen hacia el horizonte en la imagen,
    haciendo que los puntos de fuga se ubiquen cerca de esta línea.
    Para determinar las intercepciones relevantes cercanas al horizonte, se puede definir un umbral de cercanía en la imagen que se puede ajustar experimentalmente.
    Por ejemplo, en la siguiente imagen se muestran las intersecciones detectadas en la imagen original con puntos azules
    y las intersecciones relevantes con un umbral de 10 píxeles con puntos amarillos.
    \begin{figure}[!ht]
        \centering
        \includegraphics[width=0.8\textwidth]{img/reticule/relevantInter}
        \caption{Intersecciones detectadas en la imagen original}
        \label{fig:relevantInter}
    \end{figure}

    \item Algoritmo de agrupación:\\

    Para realizar la agrupación de las intersecciones relevantes, se utilizó el algoritmo \texttt{AgglomerativeClustering} de la librería \texttt{scikit-learn}.
    Este algoritmo fue elegido debido a su capacidad para manejar datos jerárquicos y su flexibilidad para determinar el número óptimo de clusters. Además,
    es robusto frente a la variabilidad en la densidad de los puntos de intersección.
    El algoritmo de agrupación jerárquica se basa en la idea de que los puntos cercanos entre sí pertenecen al mismo cluster.
    Para que el algoritmo determine el número de clusters se utilizo el parámetro \texttt{distance\_threshold} que indica la distancia máxima
    entre dos puntos para que se consideren en el mismo cluster. El valor de este parámetro se puede ajustar experimentalmente.
    Por ejemplo, en la siguiente imagen se muestran los mismos puntos de intersección relevantes del ejemplo anterior,
    pero agrupados por el algoritmo de agrupación jerárquica en 3 clusters representados de distintos colores con un símbolo \texttt{+} blanco en el centro de cada cluster.
    \begin{figure}[!ht]
        \centering
        \includegraphics[width=0.8\textwidth]{img/reticule/AgglomerativeClustering}
        \caption{Intersecciones agrupadas en clusters}
        \label{fig:clusters}
    \end{figure}
    \item Selección del punto de fuga principal:\\
%    Intersección de n líneas
%    La intersección de n líneas (bajo un criterio de mínimos cuadrados) esta dado por
%    el eigen vector asociado al eigen valor más pequeño de la matriz $M$ donde:
%    $$
%    M = \sum_{i=1}^{n} w_i l_i l_i^T
%    $$
%    donde $w_i$ es un peso asociado a la linea $l_i$ y $l_i$ es la representación homogénea de la linea $i$.

%    Kanatani, K. (1998). Statistical optimization for geometric computation: theory and practice. Elsevier.
    Pata estimar la posición del punto de fuga principal, se puede seleccionar el cluster con mayor cantidad de intersecciones y
    utilizar las lineas que generaron estos puntos para calcular la intersección de estas lineas.
    La intersección de n líneas (bajo un criterio de mínimos cuadrados) esta dado por
    el eigen vector asociado al eigen valor más pequeño de la matriz $M$ donde:
    $$ M = \sum_{i=1}^{n} w_i l_i l_i^T $$
    donde $w_i$ es un peso asociado a la linea $l_i$ y $l_i$ es la representación homogénea de la linea $i$.
%    TODO: add ref
%    Kanatani, K. (1998). Statistical optimization for geometric computation: theory and practice. Elsevier.
    De esta forma podemos conocer la posición del punto de fuga principal en el plano de la cámara.
    en la siguiente imagen se muestra el resultado de esta estimacion utilizando las lineas del cluster más grande, donde se
    representa el punto de fuga principal con un símbolo \texttt{+} amarillo.
    \begin{figure}[!ht]
        \centering
        \includegraphics[width=0.8\textwidth]{img/reticule/vanishingPoint}
        \caption{Punto de fuga principal estimado}
        \label{fig:vanishingPoint}
    \end{figure}

    \item Selección del 2do punto de fuga principal:\\

    \item Estimación de la retícula de estacionamiento:


\end{itemize}

\subsection{Experimentación y ajuste de parámetros:}\label{subsec:experimentacion-y-ajuste-de-parametros:}
Para poder determinar la configuración conveniente de los parámetros que se utilizan en los distintos algoritmos que se usan en la solución propuesta,
se realizó una aplicación en Python carga una secuencia de imágenes de la trayectoria del vehículo en el estacionamiento y por cada frame,
calcula la retícula de estacionamiento aplicando los pasos anteriormente descritos.
La aplicación permite visualizar los resultados de cada paso y ajustar dinámicamente los parámetros de los algoritmos para obtener los mejores resultados.
Los parámetros ajustables que se consideraron son:
\begin{itemize}
    \item \texttt{threshold\_image}: Umbral de binarización de la imagen.
    \item \texttt{canny\_threshold\_1}: Umbral inferior para el algoritmo de Canny.
    \item \texttt{canny\_threshold\_2}: Umbral superior para el algoritmo de Canny.
    \item \texttt{hough\_rho}: Resolución de la distancia en píxeles de la cuadrícula de la transformada de Hough.
    \item \texttt{hough\_theta}: Resolución del ángulo en radianes de la cuadrícula de la transformada de Hough.
    \item \texttt{hough\_threshold}: Umbral de votación de la transformada de Hough.
    \item \texttt{hough\_min\_line\_length}: Longitud mínima de la línea en píxeles.
    \item \texttt{hough\_max\_line\_gap}: Máxima separación entre segmentos de línea para tratarlos como una sola línea.
    \item \texttt{relevant\_intersections\_horizon\_threshold}: Umbral de cercanía al horizonte para considerar una intersección relevante.
    \item \texttt{agglomerative\_distance\_threshold}: Distancia máxima entre dos puntos para considerarlos en el mismo cluster.
\end{itemize}
Para analizar el impacto del cambio en los parámetros en tiempo real, la aplicación provee un menu de acciones que permite
representar en la imagen los resultados de cada paso del algoritmo
%bloque de codigo
\begin{lstlisting}[language=Python,label={lst:lstlisting22}]
    --- Leyenda de Teclas ---
P: Iniciar/Pausar la secuencia de imágenes.
C: Mostrar/Ocultar contornos.
L: Mostrar/Ocultar líneas.
I: Mostrar/Ocultar intersecciones.
R: Mostrar/Ocultar intersecciones relevantes.
E: Mostrar/Ocultar líneas relevantes.
A: Mostrar/Ocultar cúmulos de intersecciones.
F: Mostrar/Ocultar puntos de fuga.
G: Mostrar/Ocultar imagen binaria.
ESC: Salir.
\end{lstlisting}

A continuación se muestra un ejemplo de la aplicación en ejecución en uno de los frames de la secuencia de imágenes,
donde se visualizan los contornos, las líneas detectadas, las intersecciones, las intersecciones relevantes, las líneas relevantes,
los cúmulos de intersecciones y los puntos de fuga.

\begin{figure}[!ht]
    \begin{subfigure}{0.5\textwidth}
        \includegraphics[width=\textwidth]{img/reticule/experimentationRgb}
        \caption{Ejemplo de experimentación (RGB)}
        \label{fig:experimentationRgb}
    \end{subfigure}
    \begin{subfigure}{0.5\textwidth}
        \includegraphics[width=\textwidth]{img/reticule/experimentationBinary}
        \caption{Ejemplo de experimentación (Binaria)}
        \label{fig:experimentationBinary}
    \end{subfigure}
\end{figure}

Cómo se puede observar en la imagen, en el area de Trackbar se muestran los parámetros ajustables y su valor actual,
permitiendo al usuario modificarlos en tiempo real y observar el impacto de estos cambios en la imagen.
