\noindent
Este capítulo describe cómo se aplican en la práctica los conceptos del marco teórico 
para estimar la retícula de estacionamiento. Primero se detalla la configuración del 
entorno de simulación en CARLA (Sección \ref{sec:carla}). 
Luego se aborda el preprocesado: área de interés, umbralización, 
Canny y Hough (Sección \ref{sec:canny-hough}), seguido de la estimación de puntos de fuga 
y el filtrado de intersecciones y líneas relevantes. A partir de ello, 
se ejecuta el bucle RANSAC propuesto: muestreo de líneas por punto de fuga, cálculo de intersecciones 
(candidato a cajón), estimación de la homografía hacia un cuadrado 1×1, proyección de una retícula n×n 
y evaluación del error respecto a las líneas reales; al finalizar, se selecciona la homografía con menor 
error como representación de la retícula (Sección \ref{sec:metodo-reticula}). Por último, se presenta la 
construcción de la representación de la posición relativa y los criterios de validación experimental.