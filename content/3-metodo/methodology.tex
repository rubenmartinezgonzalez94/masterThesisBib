
Este capítulo describe la aplicación práctica de los fundamentos teóricos presentados en el Capítulo \ref{chap:marco-teorico} para estimar la retícula de estacionamiento a partir de imágenes de cámara. La metodología propuesta combina técnicas clásicas de visión por computadora con un esquema RANSAC adaptado que permite obtener una representación geométrica robusta del patrón de cajones, incluso en presencia de oclusiones, ruido y condiciones de iluminación variables.

El proceso se divide en tres etapas principales. Primero, se configura el entorno de simulación CARLA, especificando los parámetros de la cámara, la escena urbana y las condiciones de captura (sección \ref{sec:carla}).

A continuación, se realiza la detección de la retícula mediante un pipeline de procesamiento de imagen que incluye: selección del área de interés, detección de bordes con Canny, extracción de líneas candidatas con la transformada de Hough, estimación de puntos de fuga para separar las líneas en dos conjunto (horizontales y verticales) y filtrado de intersecciones relevantes (sección \ref{sec:metodo-reticula}).

Finalmente, se implementa el esquema RANSAC propuesto: en cada iteración se muestrean dos líneas de cada conjunto (asociadas a diferentes puntos de fuga), se calculan sus cuatro intersecciones como hipótesis de cajón, se estima una homografía que mapea estas esquinas a un cuadrado unitario, se proyecta una retícula $n\times n$ sobre la imagen y se evalúa el error geométrico respecto a las líneas detectadas. Tras múltiples iteraciones, se selecciona la homografía con menor error como representación final de la retícula (sección \ref{sec:metodo-ransac}).