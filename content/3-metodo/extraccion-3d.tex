Una vez estimada la homografía óptima \(\mathbf{H}\) que define la retícula de estacionamiento en el plano imagen, es necesario transformar los puntos detectados en la imagen 2D a coordenadas métricas en el espacio 3D para realizar mediciones reales de distancias y posiciones. Para ello se requiere 
aplicar la proyección perspectiva inversa utilizando los parámetros de la cámara (véase sección~\ref{sec:proyeccion-3d}).

\subsubsection{Implementación del método de intersección de rayos}

La implementación práctica de la transformación \(T^{-1}\) descrita en la ecuación~\eqref{eq:transform-T-inv} se realiza mediante la intersección de rayos con el plano del suelo.

\begin{enumerate}
    \item \textbf{Cálculo de parámetros intrínsecos:} A partir de las características de la cámara (resolución de imagen y campo de visión FOV), se calcula la distancia focal en píxeles:
    \begin{equation}
    f = \frac{w}{2 \tan(\text{FOV}/2)}
    \end{equation}
    donde \(w\) es el ancho de la imagen en píxeles. Con la distancia focal y el centro óptico 
    \((c_x, c_y)\) se construye la matriz de parámetros intrínsecos \(\mathbf{K}\).
    
    \item \textbf{Construcción de rayos de proyección:} Para cada punto \((u, v)\) detectado en la imagen (por ejemplo, extremos de líneas de la retícula), se calcula el rayo normalizado     que pasa por el centro óptico:
    \begin{equation}
    \mathbf{r}_{cam} = \mathbf{K}^{-1} \begin{bmatrix} u \\ v \\ 1 \end{bmatrix}
    \end{equation}
    
    \item \textbf{Escalado para intersección con el plano del suelo:} El rayo se escala hasta 
    que su componente $Y$ alcanza $0$ (el plano \(Y=0\) representa el suelo):
    \begin{equation}
    \lambda = \frac{h}{r_{cam,y}}
    \end{equation}
    donde \(h\) es la altura de la cámara sobre el nivel del suelo (en metros) y \(r_{cam,y}\) 
    es la componente $Y$ del rayo normalizado.
    
    \item \textbf{Obtención de coordenadas 3D en el plano del suelo:} El punto 3D resultante es:
    \begin{equation}
    \mathbf{P}_{suelo} = \lambda \cdot \mathbf{r}_{cam} = \begin{bmatrix} X \\ 0 \\ Z \end{bmatrix}
    \end{equation}
    donde \(X\) y \(Z\) son las coordenadas horizontales en metros en el plano del suelo.
\end{enumerate}

Este proceso se aplica a todos los puntos relevantes de la retícula detectada (esquinas de cajones, intersecciones de líneas, etc.), generando un conjunto de coordenadas métricas 3D que representan la estructura del estacionamiento en el espacio real.

\subsubsection{Cálculo de distancias métricas}

Una vez obtenidas las coordenadas 3D de puntos de interés \(\mathbf{P}_1 = (X_1, 0, Z_1)\) y 
\(\mathbf{P}_2 = (X_2, 0, Z_2)\), se pueden calcular distancias euclidianas reales en el plano del suelo:

\begin{equation}
d = \sqrt{(X_2 - X_1)^2 + (Z_2 - Z_1)^2}
\end{equation}

Estas distancias métricas son fundamentales para:
\begin{itemize}
    \item Determinar si un cajón tiene dimensiones suficientes para estacionar el vehículo
    \item Calcular la posición relativa del vehículo respecto a los cajones disponibles
    \item Planificar trayectorias de estacionamiento con restricciones geométricas precisas
    \item Validar que el vehículo cumpla con las dimensiones permitidas en el espacio
\end{itemize}

La Figura~\ref{fig:distances-3d-implementation} muestra un ejemplo de mediciones de distancias 
realizadas en el espacio 3D utilizando este método. Las líneas azules representan la retícula 
proyectada en el plano del suelo (vista desde arriba), las líneas rojas indican las distancias 
medidas en metros desde la posición del vehículo (rectángulo azul en el origen) hacia diferentes 
puntos de interés en la retícula. Todas las mediciones están en metros y reflejan distancias 
reales en el espacio tridimensional.

\begin{figure}[!ht]
    \centering
    \includegraphics[width=0.75\textwidth]{img/3-metodo/distances.png}
    \caption{Visualización de mediciones de distancias métricas en el espacio 3D. El rectángulo 
    azul representa la posición del vehículo en el origen, las líneas azules muestran la retícula 
    proyectada en el plano del suelo, y las líneas rojas indican distancias medidas en metros 
    desde el vehículo hacia puntos de la retícula obtenidos mediante la proyección inversa.}
    \label{fig:distances-3d-implementation}
\end{figure}
