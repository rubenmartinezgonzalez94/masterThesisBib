
Con base en los principios expuestos en la sección \ref{sec:ransac-teorico},
aplicamos una adaptación de RANSAC para estimar una homografía que haga coherente una retícula ideal con
las líneas de los cajones visibles en la imagen. Este enfoque conserva la robustez frente a outliers y
oclusiones propias del escenario de estacionamiento.

\subsection{Variación de RANSAC para ajuste de retícula:}

La variación propuesta sigue el ciclo de la Figura~\ref{fig:ramsac-flujo}:

\begin{figure}[!ht]
    \centering
    \includegraphics[width=0.99\textwidth]{img/3-metodo/ramsac-loop1.png}
    \caption{Esquema del ciclo RANSAC propuesto para el ajuste de la retícula.}
    \label{fig:ramsac-flujo}
\end{figure}


Cada iteración del ciclo comprende los siguientes pasos:
\begin{enumerate}
    \item A partir de dos puntos de fuga estimados, agrupar las líneas detectadas en dos conjuntos según su punto de fuga asociado.
    \item Muestrear aleatoriamente dos líneas de cada conjunto y calcular las cuatro intersecciones \(P_1, P_2, P_3, P_4\) (producto cruzado de pares de rectas), que definen un candidato a cajón.
    \item Estimar la homografía \(\mathbf{H}\) que mapea el cuadrado canónico de lado 1 al cuadrilátero \(\{P_k\}_{k=1}^4\).
    \item Proyectar una retícula \(n\times n\) mediante \(\mathbf{H}\) sobre la imagen.
    \item Evaluar la consistencia entre la retícula proyectada y las líneas reales (\emph{e.g.}, concordancia de orientación y proximidad en las regiones de soporte) y registrar una medida de ajuste.
    \item Almacenar temporalmente \(\mathbf{H}\) y su medida de ajuste; repetir desde el paso 2 durante \(N\) iteraciones.
\end{enumerate}


Tras completar las iteraciones, se selecciona la homografía con mejor medida de consistencia 
como representación de la retícula en la imagen. 
Esta homografía permite extender el patrón más allá del campo visible, 
infiriendo la ubicación de cajones en puntos ciegos o parcialmente ocultos, 
lo que favorece la planificación de maniobras y el estacionamiento automático. 
Además, el esquema es robusto a líneas espurias y a oclusiones, 
y tiende a ser estable entre cuadros consecutivos al reutilizar la mejor homografía previa como
punto de partida (warm-start) . En la Figura~\ref{fig:ramsac-transform} se ilustra 
la transformación de un cuadrado canónico a un cajón detectado en la imagen mediante la homografía estimada.


\begin{figure}[!ht]
    \centering
    \includegraphics[width=0.99\textwidth]{img/3-metodo/transformacion.png}
    \caption{Estimación de una homografía \(\mathbf{H}\) que mapea un cuadrado canónico (izquierda) a un cuadrilátero de un cajón detectado en la imagen (derecha), permitiendo proyectar una retícula coherente aun cuando ciertas marcas no sean visibles.}
    \label{fig:ramsac-transform}
\end{figure}

\subsection{Extracción de coordenadas en la retícula detectada}

Una vez estimada la homografía óptima \(\mathbf{H}\) mediante el esquema RANSAC descrito, 
es posible extraer información métrica y realizar mediciones en el espacio 3D del estacionamiento. 
La homografía establece una correspondencia entre puntos del plano de la retícula ideal 
(coordenadas normalizadas 2D) y puntos en el plano del suelo del mundo 3D.

Dados dos puntos en coordenadas homogéneas de la retícula normalizada:
\begin{equation}
\mathbf{p}_1 = [x_1, y_1, 1]^T, \quad \mathbf{p}_2 = [x_2, y_2, 1]^T
\end{equation}
sus correspondientes en el plano del suelo se obtienen aplicando la homografía:
\begin{equation}
\mathbf{P}_1 = \mathbf{H} \cdot \mathbf{p}_1, \quad \mathbf{P}_2 = \mathbf{H} \cdot \mathbf{p}_2
\end{equation}

Tras deshomogeneizar (dividir por la tercera coordenada), se obtienen las coordenadas 
métricas $(X_1, Y_1)$ y $(X_2, Y_2)$ en el plano del suelo. Con estas coordenadas, 
es posible calcular distancias euclidianas reales:
\begin{equation}
d = \sqrt{(X_2 - X_1)^2 + (Y_2 - Y_1)^2}
\end{equation}

Este procedimiento permite realizar mediciones precisas entre puntos de interés en la retícula, 
como las esquinas de los cajones, centros de cajones, o distancias relativas del vehículo 
respecto a la estructura del estacionamiento. La Figura~\ref{fig:distances-grid} muestra 
un ejemplo de varias distancias medidas en el espacio 3D a partir de la retícula detectada, 
donde se pueden observar las distancias desde la posición del vehículo (representado por 
el rectángulo azul en el origen) hacia diferentes puntos de la retícula proyectada.

\begin{figure}[!ht]
    \centering
    \includegraphics[width=0.6\textwidth]{img/3-metodo/distances.png}
    \caption{Ejemplo de mediciones de distancias en el espacio 3D a partir de la retícula detectada.}
    \label{fig:distances-grid}
\end{figure}
