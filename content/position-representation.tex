\noindent
Para representar la posición relativa del vehículo con respecto al espacio de estacionamiento,
se propone un sistema de coordenadas cilíndricas en el cual el origen se encuentre en la cámara del vehículo.
Esta elección provee un marco de referencia que facilita la medición de las distancias
y ángulos necesarios para maniobrar adecuadamente durante el estacionamiento.\\
\noindent
Utilizando coordenadas cilíndricas, es posible describir la posición del vehículo en términos de distancia radial, ángulo y altura,
con el origen en la cámara del vehículo permite representar de manera directa
las relaciones espaciales entre el vehículo y los límites del espacio de estacionamiento.
La altura conocida de la cámara respecto al suelo proporciona un eje de referencia constante,
mientras que los ejes orientados según la dirección y lateralidad del vehículo facilitan la interpretación y cálculo de la orientación
y desplazamiento necesarios para estacionar correctamente.\\


\subsection{Origen del sistema de coordenadas}
Para representar el origen se establecerán tres ejes coordenados:
\begin{itemize}
    \item El eje $z$ se define como la altura conocida de la cámara con respecto al suelo.
    \item El eje $x$ se define como la orientación del vehículo.
    \item El eje $y$ se define como la orientación lateral del vehículo.
\end{itemize}

\begin{figure}[!ht]
    \centering
    \begin{subfigure}{0.4\textwidth}
        \includegraphics[width=\textwidth]{img/distances_ubi_11}\label {fig:distances11}
    \end{subfigure}
    \begin{subfigure}{0.4\textwidth}
        \includegraphics[width=\textwidth]{img/distances_ubi_12}\label {fig:distances12}
    \end{subfigure}
    \caption{Origen del sistema de coordenadas.}
    \label{fig:coord}
\end{figure}
\clearpage

\subsection{Puntos de referencia}
Para representar la posición del vehículo con respecto al espacio de estacionamiento, se establecerán cuatro puntos de referencia:
\begin{itemize}
    \item El punto $1$ se define como la esquina superior izquierda del espacio de estacionamiento.
    \item El punto $2$ se define como la esquina superior derecha del espacio de estacionamiento.
    \item El punto $3$ se define como la esquina inferior izquierda del espacio de estacionamiento.
    \item El punto $4$ se define como la esquina inferior derecha del espacio de estacionamiento.
\end{itemize}
\begin{figure}[!ht]
    \centering
    \begin{subfigure}{0.4\textwidth}
        \includegraphics[width=\textwidth]{img/distances_ubi_21}\label {fig:distances21}
    \end{subfigure}
    \begin{subfigure}{0.4\textwidth}
        \includegraphics[width=\textwidth]{img/distances_ubi_22}\label {fig:distances22}
    \end{subfigure}
    \caption{Puntos de referencia.}
    \label{fig:coord2}
\end{figure}
\clearpage

\subsection{Coordenadas cilíndricas}
\noindent
Las coordenadas cilíndricas se definen como $(\rho, \theta, z)$, donde:
\begin{itemize}
    \item $\rho$ es la distancia entre el vehículo y un punto de referencia en el espacio de estacionamiento.
    \item $\theta$ es el ángulo de orientación del vehículo con respecto a un punto de referencia en el espacio de estacionamiento.
    \item $z$ es la altura de la cámara con respecto al suelo.
\end{itemize}
\begin{figure}[!ht]
    \centering
    \begin{subfigure}{0.4\textwidth}
        \includegraphics[width=\textwidth]{img/distances_ubi_31}\label {fig:distances31}
    \end{subfigure}
    \begin{subfigure}{0.4\textwidth}
        \includegraphics[width=\textwidth]{img/distances_ubi_32}\label {fig:distances32}
    \end{subfigure}
    \caption{Sistema de coordenadas cilíndricas.}
    \label{fig:coord3}
\end{figure}

\subsection{Posición relativa}
La posición relativa del vehículo con respecto al espacio de estacionamiento se representará mediante los 4 vectores de coordenadas cilíndricas:\\
\begin{center}
    $(\rho_1, \theta_1, z)$ , $(\rho_2, \theta_2, z)$ , $(\rho_3, \theta_3, z)$ , $(\rho_4, \theta_4, z)$ \\

\end{center}




