
Este trabajo presenta el desarrollo de un sistema de visión por computadora para la detección de 
retículas de estacionamiento y estimación de pose vehicular mediante procesamiento de imágenes de 
cámara. El sistema integra técnicas clásicas de visión por computadora incluyendo detección de 
bordes con Canny, extracción de líneas mediante la Transformada de Hough, y ajuste robusto de 
homografías utilizando una variación del algoritmo RANSAC adaptada específicamente para retículas 
regulares. Se implementa además un método de proyección perspectiva inversa que permite obtener 
coordenadas tridimensionales métricas a partir de puntos detectados en la imagen, proporcionando 
información geométrica precisa del entorno de estacionamiento. El desarrollo se realiza en el 
simulador CARLA, el cual proporciona un entorno de prueba realista con física vehicular de alta 
fidelidad. La validación del sistema se lleva a cabo mediante su integración como componente de 
percepción en una aplicación de aprendizaje por refuerzo para estacionamiento automático, 
demostrando que la información proporcionada por el sistema de visión es consistente para guiar decisiones de control en tiempo real. Los resultados confirman la 
viabilidad de utilizar técnicas clásicas de visión por computadora como alternativa efectiva a 
métodos exclusivamente basados en aprendizaje profundo para aplicaciones de asistencia al 
conductor y conducción autónoma.