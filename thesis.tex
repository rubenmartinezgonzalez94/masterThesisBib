\documentclass[10pt,letterpaper,final]{article}

% Paquetes
\usepackage[utf8]{inputenc}
\usepackage[spanish]{babel}
\usepackage{amsmath}
\usepackage{graphicx}
\usepackage{svg} % Para incluir gráficos SVG
\usepackage{hyperref}
\usepackage{tocbibind} % Para incluir referencias en el índice
\usepackage{caption}
\usepackage{subcaption} % Paquete para justificar el texto

% Información del documento
\title{Estimación de la pose de un vehículo mediante cámaras y sensores para estacionamiento automático en simulación}
\author{Ing. Rubén Martínez González}
\date{Septiembre 2025}
\newcommand{\university}{Universidad Autónoma de Yucatán}
\newcommand{\faculty}{Facultad de Matemáticas}
\newcommand{\advisor}{Dr. Arturo Espinosa Romero}
\newcommand{\myauthor}{Ing. Rubén Martínez González}
\newcommand{\degree}{Maestría en Ciencias de la Computación}
\newcommand{\tesisInv}{Tesis de investigación}
\newcommand{\mydate}{Septiembre 2025}
\newcommand{\mytitle}{Estimación de la pose de un vehículo mediante cámaras y sensores para estacionamiento automático en simulación}

\begin{document}

% Portada moderna
\begin{titlepage}
    \centering
    % Logos
    \begin{minipage}{0.25\textwidth}
        \centering
        \includegraphics[width=0.8\textwidth]{img/fmat.png}
    \end{minipage}%
    \hfill
    \begin{minipage}{0.5\textwidth}
        \centering
        {\scshape\small \university \par}
        \vspace{0.2cm}
        {\scshape\small \faculty \par}
        \vspace{0.2cm}
        {\scshape\small \degree \par}
    \end{minipage}%
    \hfill
    \begin{minipage}{0.25\textwidth}
        \centering
        \includegraphics[width=0.8\textwidth]{img/UADY.jpg}
    \end{minipage}

    \vspace{1.5cm}
    {\LARGE\bfseries \mytitle \par}
    \vspace{1.5cm}
    {\Large \tesisInv \par}
    \vspace{1cm}
    {\large Autor: \textbf{\myauthor}\par}
    \vspace{0.5cm}
    {\large Asesor: \textbf{\advisor}\par}
    \vfill
    \vspace{0.5cm}
    {\large \mydate \par}
\end{titlepage}

% Resumen
\begin{abstract}
    \noindent
Este documento presenta un enfoque para la estimación de la posición de un vehículo en un entorno de estacionamiento
utilizando técnicas de visión por computadora y aprendizaje automático. Se describe la implementación de un sistema
de simulación que permite la recolección de datos de sensores y la generación de imágenes sintéticas para entrenar 
modelos de detección de objetos. Los resultados muestran la efectividad del enfoque propuesto en la identificación de 
espacios de estacionamiento y la navegación autónoma en entornos complejos.
\end{abstract}
\clearpage


% Índice
\tableofcontents
\listoffigures
\clearpage

% Capítulo 1: Introducción
\section{Introducción}

El estacionamiento es una actividad esencial en la vida urbana, ya que permite a los conductores dejar sus vehículos en reposo mientras realizan otras actividades.
No obstante, el estacionamiento en áreas urbanas congestionadas presenta múltiples desafíos, incluyendo la falta de espacio,
la presencia de obstáculos y la visibilidad reducida, lo cual incrementa el riesgo de colisiones y daños a los vehículos.
Incluso habiendo identificado el espacio de estacionamiento, el proceso de maniobrar el vehículo para estacionar puede ser complicado y estresante,
especialmente para conductores con poca experiencia o en vehículos grandes.\\

Para mitigar estos problemas, se han desarrollado sistemas avanzados de asistencia al conductor, como el estacionamiento automático,
que facilitan esta tarea y mejoran la seguridad.
Sin embargo, la efectividad de estos sistemas depende en gran medida de la capacidad para estimar con precisión la posición del vehículo
con respecto al espacio de estacionamiento.
Una estimación incorrecta puede resultar en maniobras inseguras, especialmente en entornos con espacio limitado.\\

Representar esta ubicación de manera adecuada es crucial para el desarrollo de un sistema de estacionamiento automático confiable.
Para abordar esta cuestión, se propone diseñar una simulación en donde se obtendrán datos de sensores
y cámaras del vehículo para estimar su posición relativa al espacio de estacionamiento.\\

La simulación se llevará a cabo en un entorno controlado que refleja las condiciones reales de estacionamiento.
%Se modelará un escenario de estacionamiento con un vehículo y su espacio de estacionamiento, donde se adquirirán datos de sensores.
Dicho entorno consistirá de un cajón de estacionamiento objetivo, el vehículo que se va a controlar y objetos estáticos que pueden generar oclusiones.
El entorno de simulación permitirá extraer información a través de sensores los cuales permitirán hacer mediciones de las distancias y ángulos necesarios para maniobrar durante el estacionamiento.\\

La información geométrica extraída mediante técnicas de visión por computadora permitirá conocer la posición y orientación del vehículo respecto a la estructura del estacionamiento, proporcionando mediciones métricas reales del entorno tridimensional.
Esta información constituye un componente fundamental que puede ser utilizado por diversos sistemas de control y planificación de trayectorias en aplicaciones de conducción autónoma.\\

La investigación se centra en desarrollar un sistema robusto de visión por computadora que, mediante el procesamiento de imágenes de cámara y la reconstrucción geométrica del entorno de estacionamiento, proporcione información precisa de pose vehicular aplicable a sistemas de asistencia al conductor y estacionamiento automático en entornos urbanos.

\subsection{Contexto y problemática}
%    problemas en los estacionamientos
Los estacionamientos son imprescindibles en la vía urbana, ya que permiten a los conductores estacionar sus vehículos
de manera segura y eficiente. Sin embargo, el proceso de estacionamiento puede ser complicado y estresante,
especialmente en áreas congestionadas con espacio limitado y visibilidad reducida.
Factores como la falta de espacio, la presencia de obstáculos y la poca visibilidad para el conductor ocasionan
dificultades al estacionar un vehículo, lo que puede aumentar el riesgo de colisiones y daños al vehículo.
\\
%   se han logrado sistemas de asistencia al conductor
En la actualidad, la búsqueda de soluciones para mejorar la eficiencia y seguridad en el desplazamiento vehicular
ha llevado al desarrollo de sistemas avanzados de asistencia al conductor.
Entre estos sistemas, el estacionamiento automático ha ganado relevancia como una función que puede contribuir
a reducir los riesgos asociados con el estacionamiento en entornos urbanos congestionados.
\\
%   problemas de sistemas de estacionamiento automático
Sin embargo, el desarrollo de sistemas de estacionamiento automático presenta desafíos significativos,
especialmente en lo que respecta a la estimación de la posición del vehículo con respecto al espacio de estacionamiento.
El cálculo incorrecto de esta posición puede resultar en maniobras de estacionamiento inseguras o peligrosas,
especialmente en entornos donde el espacio de estacionamiento es limitado o con poca visibilidad para el conductor.
\\
%   necesidad de sistemas de asistencia al conductor más autónomos
En este contexto, continua la necesidad de desarrollar soluciones de utilidad para que los sistemas de asistencia al conductor
sean cada vez más autosuficientes y no dependan de la intervención limitada del conductor.
\\
%   en que consiste la investigación
Esta investigación se enfoca en desarrollar un sistema de visión por computadora para la detección 
automática de retículas de estacionamiento y la estimación precisa de la pose del vehículo mediante 
el procesamiento de imágenes de cámara, proporcionando información geométrica métrica que pueda ser 
utilizada por sistemas de asistencia al conductor y aplicaciones de conducción autónoma.

\subsection{Preguntas de investigación}
\noindent
\begin{itemize}
    \item ¿Cómo se puede representar la posición de un vehículo con respecto a su espacio de estacionamiento?
    \item ¿Cómo se puede estimar esta posición utilizando las cámaras y sensores del vehículo?
    \item ¿Cómo usar esta posición estimada para que el vehículo se estacione automáticamente?
\end{itemize}

\subsection{Hipótesis}

``Mediante técnicas de visión por computadora aplicadas al procesamiento de imágenes de cámara,
es posible desarrollar un sistema de detección de retículas de estacionamiento y estimación 
de pose vehicular que proporcione información geométrica precisa utilizable en aplicaciones de 
conducción autónoma.''


\subsection{Objetivos}
\subsubsection{Objetivo General}

Desarrollar un sistema de estimación de la posición relativa al estacionamiento de un vehículo mediante cámaras y sensores para estacionamiento automático.

\subsubsection{Objetivos específicos}

\begin{itemize}
    \item Modelar un ambiente de simulación de un vehículo y estacionamiento.
    \item Obtener datos de los sensores del vehículo en simulación.
    \item Interpretar los datos de los sensores mediante técnicas de visión computacional.
    \item Procesar los datos y estimar la posición del vehículo con respecto al estacionamiento.
    \item Utilizar la posición estimada para lograr un sistema de estacionamiento automático en simulación.
\end{itemize}

\subsection{Estado del arte - Trabajos previos relacionados}
\noindent\textbf{Autonomous Driving Architectures: Insights of Machine Learning and Deep Learning Algorithms}~\cite{bachute2021autonomous}
El artículo fue publicado en la revista Machine Learning with Applications en 2021 y
proporciona una visión general de la aplicación de algoritmos de Aprendizaje Automático y Aprendizaje Profundo
en sistemas de conducción autónoma, destacando su evaluación en tareas cruciales.
Se destaca el creciente impulso en la investigación de la conducción autónoma debido a sus ventajas inherentes, como la reducción
de la intervención humana y la disociación del conductor del vehículo.
Se subraya la complejidad de estos sistemas,
que involucra la integración de múltiples subsistemas, y se analizan diversas tareas específicas dentro de la conducción autónoma,
como la planificación de movimiento, la detección de peatones y señales de tráfico, el estacionamiento automatizado, entre otras.
El estudio se centra en la aplicación de algoritmos de Aprendizaje Automático y Aprendizaje Profundo para abordar estas tareas,
evaluando y comparando su rendimiento a través de métricas específicas. La investigación ofrece una perspectiva amplia sobre el uso
y la evaluación en el contexto de la conducción autónoma.
\begin{figure}[!ht]
    \centering
    \begin{subfigure}{0.4\textwidth}
        \includegraphics[width=\textwidth]{img/12Screenshot_20231106_142954}\label{fig:12}
    \end{subfigure}
    \begin{subfigure}{0.4\textwidth}
        \includegraphics[width=\textwidth]{img/13Screenshot_20231106_143018}\label{fig:13}
    \end{subfigure}
    \begin{subfigure}{0.4\textwidth}
        \includegraphics[width=\textwidth]{img/15Screenshot_20231106_143633}\label{fig:15}
    \end{subfigure}
    \begin{subfigure}{0.5\textwidth}
        \includegraphics[width=\textwidth]{img/14 Screenshot_20231106_143419}\label{fig:14}
    \end{subfigure}
    \begin{subfigure}{\textwidth}
        \includegraphics[width=0.99\textwidth]{img/16Screenshot_20231106_143701}\label{fig:16}
    \end{subfigure}
\end{figure}
\clearpage

\noindent\textbf{Vision-based autonomous car racing using deep imitative reinforcement learning}~\cite{cai2021vision}
El artículo fue publicado en la revista IEEE Robotics and Automation Letters en 2021 y aborda el desafío del automovilismo autónomo
en el campo del control robótico, históricamente dependiente de mapas precisos, localización y planificación, lo que lo hace
computacionalmente ineficiente y sensible a cambios en el entorno.
\\
Se destaca el desarrollo de sistemas de aprendizaje profundo de extremo a extremo, que muestran resultados prometedores en la conducción
autónoma.
\\
Sin embargo, estos sistemas suelen basarse en aprendizaje por imitación supervisada (IL), enfrentando problemas de discrepancia
en la distribución de datos.
\\
Aunque se han empleado métodos de aprendizaje por refuerzo (RL), requieren grandes cantidades de datos de interacción riesgosa.
\\
El artículo presenta un enfoque innovador denominado aprendizaje profundo imitativo y de refuerzo (DIRL), que logra la agilidad en
el automovilismo autónomo mediante el uso de entradas visuales.
\\
Este enfoque combina el conocimiento adquirido tanto del aprendizaje por imitación como del aprendizaje basado en modelos de RL,
permitiendo al agente aprender de instructores humanos y mejorar su rendimiento interactuando con un modelo de mundo offline.
La validación del algoritmo se lleva a cabo tanto en simulaciones de conducción de alta fidelidad como en un automóvil RC a escala 1/20 en
el mundo real, con capacidad computacional limitada. \\
Los resultados de la evaluación demuestran que este método supera a enfoques anteriores de IL y RL en eficiencia de muestra y rendimiento
en la tarea, mostrando un gran potencial en el ámbito de la conducción autónoma.
\begin{figure}[!ht]
    \centering
    \begin{subfigure}{0.4\textwidth}
        \includegraphics[width=\textwidth]{img/21}\label{fig:21}
    \end{subfigure}
    \begin{subfigure}{0.4\textwidth}
        \includegraphics[width=\textwidth]{img/22}\label{fig:22}
    \end{subfigure}
%            \begin{subfigure}
%                \includegraphics[width=0.2\textwidth]{img/23}\label{fig:23}
%            \end{subfigure}
\end{figure}

\clearpage

\noindent\textbf{Model-based probabilistic collision detection in autonomous driving}~\cite{althoff2009model}
El artículo fue publicado en la revista IEEE Transactions on Intelligent Transportation Systems en 2009 y se centra en la seguridad vial
de los vehículos autónomos en entornos de tráfico complejo.
Su enfoque principal es la detección probabilística de colisiones mediante el análisis y la predicción de la ocupación de la carretera
por parte de otros vehículos.\\
El estudio aborda la incertidumbre inherente en la interacción entre los vehículos autónomos y otros actores del tráfico.
Analiza cómo las mediciones y los posibles comportamientos de estos afectan la predicción de posibles colisiones.
Además, considera las limitaciones en las maniobras de conducción debidas a la geometría de la carretera y la influencia
de estas restricciones en la probabilidad de colisión para trayectorias específicas.\\
Lo más destacado de este enfoque es su eficiencia. La mayor parte de los cálculos intensivos se llevan a cabo offline,
permitiendo disponer de un algoritmo en línea eficiente para aplicaciones en tiempo real.                                                                           \\
Esto contribuye significativamente a la seguridad vial al proporcionar una herramienta precisa y eficaz para la detección anticipada de
posibles colisiones en entornos de conducción autónoma.                                                            \\
\begin{figure}[!ht]
    \begin{subfigure}{0.4\textwidth}
        \includegraphics[width=\textwidth]{img/31}\label{fig:31}
    \end{subfigure}
    \begin{subfigure}{0.4\textwidth}
        \includegraphics[width=\textwidth]{img/35}\label{fig:35}
    \end{subfigure}
%            \begin{subfigure}
%                \includegraphics[width=0.5\textwidth]{img/32}\label{fig:32}
%            \end{subfigure}
    \vspace{2cm}
    \begin{subfigure}{0.4\textwidth}
        \includegraphics[width=\textwidth]{img/33}\label{fig:33}
    \end{subfigure}
    \begin{subfigure}{0.4\textwidth}
        \includegraphics[width=\textwidth]{img/34}\label{fig:34}
    \end{subfigure}

\end{figure}
\clearpage

\noindent\textbf{Vision-based autonomous vehicle systems based on deep learning: A systematic literature review}~\cite{pavel2022vision}
El artículo fue publicado en la revista Applied Science en 2022 y presenta una revisión sistemática de la literatura sobre el empleo
de técnicas de aprendizaje profundo en los sistemas
de vehículos autónomos a lo largo de la última década.                                                                    \\Esta revisión se divide en varios módulos que abarcan distintos aspectos,
desde el análisis de percepción y la toma de decisiones hasta el control, la planificación de trayectorias
y la visualización en sistemas de realidad aumentada tipo HUD.                                                                                                                   \\
Se examinan investigaciones llevadas a cabo entre 2011 y 2021 que se enfocan en la utilización de cámaras RGB como sensores principales
en estos sistemas. Se otorga especial atención a los resultados finales, destacando la visualización en sistemas de realidad aumentada
basados en HUD.                                                                                                      \\Esto incluye advertencias tempranas, marcadores en la carretera para mejorar la navegación y la seguridad, superposición
de información en vehículos y peatones en condiciones visuales extremas para reducir colisiones.
La revisión subraya los métodos actuales de aprendizaje profundo que se basan únicamente en la visión de cámaras RGB, prescindiendo de la
compleja fusión de sensores.                                                                     \\Se espera que este enfoque allane el camino para el desarrollo ágil de sistemas de vehículos autónomos,
siendo prácticos, eficientes y seguros en términos de costos.                                                               \\
\begin{figure}[!ht]
    \centering
    \begin{subfigure}{\textwidth}
        \includegraphics[width=1\textwidth]{img/71}\label{fig:71}
    \end{subfigure}
%            \begin{subfigure}
%                \includegraphics[width=0.5\textwidth]{img/72}\label{fig:72}
%            \end{subfigure}
    \begin{subfigure}{0.4\textwidth}
        \includegraphics[width=\textwidth]{img/73}\label{fig:73}
    \end{subfigure}
    \begin{subfigure}{0.4\textwidth}
        \includegraphics[width=\textwidth]{img/74}\label{fig:74}
    \end{subfigure}
\end{figure}
\clearpage

\noindent\textbf{A cost-effective computer vision-based vehicle detection system}~\cite{alam2022cost}
El artículo fue publicado en la revista Concurrent Engineering en 2022 y se enfoca en la detección de vehículos.
\\Destaca la importancia crítica del procesamiento rápido y la detección precisa de vehículos dentro de un sistema autónomo de detección.
\\Presenta un sistema de detección de vehículos basado en visión por computadora que utiliza un algoritmo de Gentle Adaptive Boosting
con características tipo Haar para generar hipótesis de vehículos de manera eficiente.
Para abordar los errores potenciales, propone el uso de un algoritmo de Máquinas de Vectores de Soporte (SVM) entrenado con características
del histograma de gradientes orientados (HOG) para filtrar las hipótesis falsas.
\\El descriptor HOG se centra en la forma y contornos de los vehículos, mejorando la precisión de la detección.
La combinación de características tipo Haar y HOG permite cumplir los objetivos de detección en la conducción autónoma.
\\El rendimiento del sistema propuesto se evalúa con imágenes capturadas durante el día y la noche y se compara con tres detectores
de vehículos existentes. Los resultados muestran una precisión promedio del 0.97 para imágenes capturadas durante el día
y del 0.94 para imágenes nocturnas.                                                                                               \\Además, se destaca que el sistema propuesto requiere aproximadamente 15 veces menos tiempo
de entrenamiento en comparación con las técnicas existentes, utilizando la misma cantidad de datos de imágenes y la misma unidad
de procesamiento central (CPU). Esto demuestra una mejora significativa en la eficiencia del sistema propuesto en términos de tiempo de entrenamiento.
\begin{figure}[!ht]
\centering
%            \begin{subfigure}
%                \includegraphics[width=0.6\textwidth]{img/81}\label{fig:81}
%            \end{subfigure}
%            \begin{subfigure}
%                \includegraphics[width=0.5\textwidth]{img/86}\label{fig:82}
%            \end{subfigure}
    \begin{subfigure}{0.4\textwidth}
        \includegraphics[width=\textwidth]{img/84}\label{fig:84}
    \end{subfigure}
    \begin{subfigure}{0.4\textwidth}
        \includegraphics[width=\textwidth]{img/82}\label{fig:86}
    \end{subfigure}
\end{figure}
\clearpage


% \subsubsection{Tabla comparativa}
% \begin{center}
    \resizebox{\textwidth}{!}{
        \begin{tabular}{|p{5cm}|p{2cm}|p{2cm}|p{2cm}|p{2cm}|p{2cm}|p{2cm}|}
            \hline
            \textbf{Características}
            & \textbf{Propia}
            & \textbf{Autonomous Driving Architectures \cite{bachute2021autonomous}}
            & \textbf{Vision-based Autonomous Car Racing \cite{cai2021vision}}
            & \textbf{Model-based Probabilistic Collision Detection \cite{althoff2009model}}
            & \textbf{Vision-based Autonomous Vehicle Systems \cite{pavel2022vision}}
            & \textbf{Cost-effective Vehicle Detection System \cite{alam2022cost}} \\
            \hline
            Uso de algoritmos de Aprendizaje Automático y Aprendizaje Profundo & X & X &   &   & X &   \\
            \hline
            Enfoque en la conducción autónoma                                  & X & X & X & X & X & X \\
            \hline
            Ventajas de la conducción autónoma                                 & X & X &   &   &   &   \\
            \hline
            Complejidad de los sistemas de conducción autónoma                 &   & X &   &   &   &   \\
            \hline
            Análisis de tareas en la conducción autónoma                       & X & X &   &   &   &   \\
            \hline
            Evaluación y comparación de algoritmos                             &   & X & X &   &   &   \\
            \hline
            Predicción estocástica de ocupación de la carretera                &   &   &   & X &   &   \\
            \hline
            Eficiencia en cálculos intensivos                                  &   &   & X & X &   &   \\
            \hline
            Utilización de cámaras RGB como sensores principales               & X &   & X &   & X &   \\
            \hline
            Detección de vehículos en conducción autónoma                      & X &   &   &   &   & X \\
            \hline
        \end{tabular}
    }
\end{center}

% Capítulo 2: Marco teórico
\clearpage
\section{Marco teórico}\label{chap:marco-teorico}
Este capítulo reúne los fundamentos teóricos y computacionales necesarios para comprender la metodología propuesta. Se inicia con la formación de imagen y el modelo de cámara pin-hole, explicando de manera intuitiva la proyección de puntos y líneas, así como la representación homogénea para operaciones geométricas (Sección \ref{subsec:camera}).

A continuación, se abordan las herramientas matemáticas para describir y manipular líneas y puntos en el plano imagen, incluyendo la obtención de líneas a partir de puntos, intersección de líneas, y el ajuste óptimo mediante formulaciones basadas en el espacio nulo, siguiendo la teoría de Kanatani y Hartley \& Zisserman.

Se presentan las homografías para el plano del suelo (Sección \ref{subsec:homografias}), y las técnicas de detección de bordes y líneas empleadas, en particular el método de Canny y la transformada de Hough (Sección \ref{sec:canny-hough}).

La sección central describe el algoritmo RANSAC para el ajuste robusto de modelos en presencia de outliers, detallando su funcionamiento y parámetros clave según la referencia clásica de Fischler y Bolles y la exposición de Hartley \& Zisserman (Sección \ref{sec:ransac-teorico}).

Finalmente, se establecen las convenciones de marcos de referencia y representación de la pose, y se describen las plataformas y bibliotecas utilizadas, como CARLA y OpenCV (Sección \ref{sec:plataformas}).


\subsection{Plataformas y librerías}\label{sec:plataformas}

La simulación computacional se ha consolidado como una herramienta fundamental
en el desarrollo y validación de sistemas de conducción autónoma.
Permite recrear escenarios complejos y potencialmente peligrosos de manera segura,
flexible y económica, facilitando la experimentación y el análisis de algoritmos
antes de su implementación en vehículos reales. En el contexto de este trabajo,
la simulación resulta especialmente útil para modelar situaciones de estacionamiento
automático, donde la precisión y la seguridad son críticas. A través de la simulación,
es posible ajustar parámetros, evaluar el desempeño de sensores virtuales y analizar el
comportamiento del sistema bajo diferentes condiciones ambientales y de tráfico,
todo ello sin los riesgos y costos asociados a las pruebas físicas.
La simulación puede proporcionar información detallada sobre el vehículo y
su entorno, lo que permite realizar mediciones y análisis exhaustivos de los
datos obtenidos, de manera similar a como se haría en la vida real.

En este trabajo se emplean plataformas y bibliotecas ampliamente adoptadas en visión
por computadora y simulación para vehículos autónomos.
En particular, utilizamos CARLA, un simulador de código abierto que permite recrear entornos urbanos realistas
con agentes dinámicos y sensores virtuales configurables, y OpenCV, una biblioteca de procesamiento
de imágenes que proporciona las operaciones básicas para umbralización, detección de bordes (Canny)
y detección de líneas (Hough), entre otras.

\subsection{CARLA Simulator}\label{sec:carla-teorico}

CARLA (Car Learning to Act) es una plataforma orientada a investigación que facilita la generación de escenas
complejas y controladas, con variabilidad de clima, iluminación y tráfico, y un conjunto de sensores virtuales
(cámaras, LiDAR, radar, GPS) que producen datos cercanos a los de un vehículo real.
Su flexibilidad para instrumentar escenarios y capturar datos reproducibles lo hace idóneo para
experimentación y validación de algoritmos de percepción, localización y control en contextos como el
estacionamiento automático \cite{dosovitskiy2017carla}.
En la Sección \ref{subsec:simulation-design} se detalla el diseño específico del entorno de simulación que
utilizamos, y en la Figura~\ref{fig:carla-simulator-teo} (véase Sección \ref{sec:carla-teorico}) se ilustran ejemplos
de condiciones ambientales configurables en la plataforma.

\begin{figure}[!ht]
	\centering
	\includegraphics[width=0.8\textwidth]{img/carla_clima_example}
	\caption{Ejemplos de condiciones ambientales configurables en el simulador CARLA.}
	\label{fig:carla-simulator-teo}
\end{figure}

\subsection{OpenCV}

OpenCV proporciona el conjunto de operaciones y estructuras necesarias para el preprocesado de imágenes y la extracción de primitivas geométricas, incluyendo umbralización, Canny y la transformada de Hough, que se emplean en la detección de la retícula de estacionamiento.

\subsection{scikit-learn: \texttt{AgglomerativeClustering}}\label{sec:sklearn-agglomerative}

Para la agrupación de intersecciones utilizamos el algoritmo de \emph{clustering} jerárquico aglomerativo provisto por \texttt{scikit-learn}.
Este enfoque construye una jerarquía de fusiones de abajo hacia arriba, uniendo en cada paso los grupos más próximos según
una métrica de distancia y un criterio de enlace (\texttt{linkage} en \{\texttt{ward}, \texttt{complete}, \texttt{average}, \texttt{single}\}).
El parámetro \texttt{distance\_threshold} permite dejar que el propio algoritmo determine el número de clusters deteniendo
las fusiones cuando la distancia supera un umbral, lo cual resulta práctico en escenas con variabilidad de densidad.
Esta técnica es apropiada para consolidar intersecciones cercanas al horizonte en cúmulos cuya localización (centroide)
sirve de estimación inicial de puntos de fuga \cite{tan2005introduction}.


\subsection{Herramienta de experimentación y ajuste de parámetros}\label{sec:experimentation-tool}

Para determinar la configuración óptima de parámetros de la solución propuesta, se desarrolló una aplicación en Python
que carga una secuencia de imágenes de la trayectoria del vehículo en el estacionamiento y, para cada cuadro,
calcula la retícula aplicando los pasos descritos en este trabajo. La herramienta permite visualizar el resultado de cada
etapa y ajustar dinámicamente los parámetros para analizar su impacto en tiempo real.


Parámetros ajustables considerados:
\begin{itemize}
	\item \texttt{threshold\_image}: Umbral de binarización de la imagen.
	\item \texttt{canny\_threshold\_1}: Umbral inferior para Canny.
	\item \texttt{canny\_threshold\_2}: Umbral superior para Canny.
	\item \texttt{hough\_rho}: Resolución de distancia (píxeles) en Hough.
	\item \texttt{hough\_theta}: Resolución angular (radianes) en Hough.
	\item \texttt{hough\_threshold}: Umbral de votos en Hough.
	\item \texttt{hough\_min\_line\_length}: Longitud mínima de segmento.
	\item \texttt{hough\_max\_line\_gap}: Máxima separación para unir segmentos.
	\item \texttt{relevant\_intersections\_horizon\_threshold}: Umbral de cercanía al horizonte.
	\item \texttt{agglomerative\_distance\_threshold}: Distancia máxima para pertenecer al mismo cúmulo.
\end{itemize}


A modo ilustrativo, a continuación  en la figura \ref{fig:experimentationBinary-teo} 
se muestra la herramienta donde se visualizan contornos, líneas detectadas,
intersecciones, intersecciones relevantes, cúmulos y puntos de fuga.

\begin{figure}[!ht]
	% \begin{subfigure}{0.5\textwidth}
	% 	\includegraphics[width=\textwidth]{img/reticule/experimentationRgb}
	% 	\caption{Ejemplo de experimentación (RGB)}
	% 	\label{fig:experimentationRgb-teo}
	% \end{subfigure}
	
		\includegraphics[width=0.8\textwidth]{img/reticule/experimentationBinary}
		\caption{Ejemplo de experimentación (Binaria)}
		\label{fig:experimentationBinary-teo}
	
\end{figure}


La interfaz incorpora una "Trackbar" para modificar parámetros en tiempo real y observar su efecto inmediato,
facilitando la búsqueda de configuraciones estables y robustas para distintas condiciones de escena.

\subsection{Gymnasium}\label{sec:gymnasium}

Gymnasium es la biblioteca estándar para entornos de aprendizaje por refuerzo que proporciona una interfaz
unificada entre agentes y entornos \cite{towers2024gymnasium}. Sus principales ventajas incluyen:
compatibilidad con múltiples algoritmos de RL, estandarización de espacios de acción y observación,
y un ecosistema extenso de entornos preconfigurados. Los \texttt{Wrappers} permiten modificar el comportamiento
de entornos existentes sin alterar su código base, facilitando adaptaciones como normalización de observaciones,
transformación de espacios de acción, o integración con simuladores externos.

\subsection{Stable-Baselines3}\label{sec:stable-baselines3}

Stable-Baselines3 (SB3) es una biblioteca que implementa algoritmos de aprendizaje por refuerzo de última
generación con código confiable y bien documentado \cite{raffin2021stable}. Incluye algoritmos como
PPO (Proximal Policy Optimization), SAC (Soft Actor-Critic), TD3 (Twin Delayed Deep Deterministic Policy Gradient),
y soporte para técnicas avanzadas como HER (Hindsight Experience Replay). Su diseño modular permite
experimentación rápida con diferentes algoritmos manteniendo consistencia en hiperparámetros y evaluación.


\subsection{RANSAC para homografías basadas en líneas}\label{sec:ransac-teorico}
\noindent
RANSAC (Random Sample Consensus) es un paradigma de estimación robusta de modelos que alterna 
muestreos aleatorios mínimos con una evaluación de consenso (inliers) para encontrar hipótesis 
consistentes aun en presencia de una proporción elevada de datos atípicos. 
En cada iteración se selecciona un subconjunto mínimo de observaciones, se ajusta un modelo candidato 
y se mide cuántas observaciones concuerdan con él dentro de una tolerancia; tras múltiples iteraciones, 
se elige la hipótesis con mayor soporte y, opcionalmente, se refina con los inliers. 
Esta estrategia, ampliamente utilizada en visión por computadora y cartografía automatizada, 
fue introducida por Fischler y Bolles \cite{fischler1981ransac}.

\noindent
Beneficios clave del enfoque RANSAC:
\begin{itemize}
	\item Tolerancia a outliers y ruido: puede recuperar el modelo correcto aunque una parte sustancial de las observaciones sea espuria u ocluida.
	\item Flexibilidad de modelo: se aplica a múltiples familias (rectas, homografías, transformaciones, etc.).
	\item Sencillez operativa: alterna muestreo mínimo, ajuste y conteo de consenso.
\end{itemize}

\noindent
Variación propuesta en este trabajo. 
Como se explicará con más detalle en la Sección \ref{sec:metodo-ransac}, 
en esta investigación adaptamos RANSAC para estimar una homografía que alinee una retícula ideal 
con las líneas de los cajones de estacionamiento observadas en la imagen. 
Bajo las siguientes consideraciones generales: 
(i) se emplean dos puntos de fuga para agrupar las líneas en dos conjuntos según su punto de fuga; 
(ii) la muestra mínima se construye eligiendo dos líneas de cada conjunto (uno por cada punto de fuga) para formar cuatro intersecciones 
que hipotetizan un cajón; y 
(iii) a partir de esas esquinas se estima una homografía y se extiende una retícula \(n\times n\) 
sobre la imagen. Este planteamiento proporciona coherencia geométrica global del patrón 
y permite extrapolar la retícula más allá del campo visible, infiriendo la ubicación de cajones 
en puntos ciegos o parcialmente ocultos, lo cual resulta útil para la planificación 
y el estacionamiento automático.


% Capítulo 3: Metodología
\clearpage
\section{Metodología}\label{chap:metodologia}
\noindent
Este capítulo describe cómo se aplican en la práctica los conceptos del marco teórico 
para estimar la retícula de estacionamiento. Primero se detalla la configuración del 
entorno de simulación en CARLA (Sección \ref{sec:carla}). 
Luego se aborda el preprocesado: área de interés, umbralización, 
Canny y Hough (Sección \ref{sec:canny-hough}), seguido de la estimación de puntos de fuga 
y el filtrado de intersecciones y líneas relevantes. A partir de ello, 
se ejecuta el bucle RANSAC propuesto: muestreo de líneas por punto de fuga, cálculo de intersecciones 
(candidato a cajón), estimación de la homografía hacia un cuadrado 1×1, proyección de una retícula n×n 
y evaluación del error respecto a las líneas reales; al finalizar, se selecciona la homografía con menor 
error como representación de la retícula (Sección \ref{sec:metodo-reticula}). Por último, se presenta la 
construcción de la representación de la posición relativa y los criterios de validación experimental.

\subsection{Entorno de simulación}\label{sec:carla}

De acuerdo con lo expuesto en la sección \ref{sec:plataformas},
empleamos CARLA como entorno de simulación para nuestros experimentos.
El escenario de estacionamiento incluye un vehículo con el objetivo de estacionarse y un área con cajones marcados.
La posición inicial del vehículo se varía de forma controlada entre ejecuciones.
La cámara frontal se monta en la zona del retrovisor a altura conocida,
y se recopilan secuencias de imágenes para alimentar el pipeline de detección de líneas,
estimación de puntos de fuga y ajuste de retícula por RANSAC.
La Figura~\ref{fig:simulation-design} muestra un ejemplo del diseño del entorno.

\begin{figure}[!ht]
    \centering
    \begin{subfigure}{0.4\textwidth}
        \includegraphics[width=\textwidth]{img/distances}\label {fig:distances}
    \end{subfigure}
    \begin{subfigure}{0.4\textwidth}
        \includegraphics[width=\textwidth]{img/distances2}\label {fig:distances2}
    \end{subfigure}

    \caption{Diseño del entorno de simulación en CARLA.}
    \label{fig:simulation-design}
\end{figure}


La Figura~\ref{fig:camera-view} ilustra ejemplos de la vista de la cámara frontal desde la ubicación definida.

\begin{figure}[!ht]
    \centering
    \begin{subfigure}{0.4\textwidth}
        \includegraphics[width=\textwidth]{img/mirrow_camara_ex}\label {fig:camara}
    \end{subfigure}
    \begin{subfigure}{0.4\textwidth}
        \includegraphics[width=\textwidth]{img/mirrow_camara_ex2}\label {fig:camara2}
    \end{subfigure}
    \caption{Vista de la cámara en el entorno de simulación.}
    \label{fig:camera-view}
\end{figure}






\subsection{Detección de la retícula de estacionamiento}\label{sec:metodo-reticula}
\noindent
En esta sección desarrollamos el procedimiento práctico para detectar la
retícula de estacionamiento a partir de una imagen de la cámara frontal.
Nos apoyamos en los fundamentos del Marco teórico: Canny y Hough (Sección~\ref{sec:canny-hough}),
la representación de rectas y su intersección en coordenadas homogéneas
(Sección~\ref{sec:rectas-svd}), y los puntos de fuga (Sección~\ref{sec:vanishing-points}).
\noindent
Partimos de una imagen RGB capturada por la cámara frontal. Trabajamos sobre una región de interés (ROI)
definida por debajo del horizonte excluyendo la franja superior para reducir ruido.
Como resultado del procesamiento, obtenemos conjuntos de líneas agrupados según su punto de fuga y un conjunto de esquinas candidatas a vértices de cajón. Dada la naturaleza de la escena, pueden presentarse dificultades como baja textura o contraste, oclusiones y marcas del pavimento degradadas; para mitigarlas, empleamos umbrales ajustables y reintentos por cuadro.

\noindent
Por la proyección perspectiva, las líneas paralelas del mundo se intersectan en la imagen en un punto de fuga
(véase Sección~\ref{sec:vanishing-points} y Fig.~\ref{fig:distorion-teo}).
En nuestro contexto, las marcas de los cajones son aproximadamente paralelas y generan
una retícula de paralelogramos; esto habilita usar la detección de líneas para identificar puntos de fuga
y estimar la posición de la retícula.


\begin{figure}[!ht]
    \centering
    \includegraphics[width=0.8\textwidth]{img/3-metodo/piperline-reticule.png}
    \caption{Diagrama general del pipeline para la detección de la retícula de estacionamiento.}
    \label{fig:reticula-pipeline}
\end{figure}

\noindent
A continuación, desarrollamos cada etapa del diagrama de la Figura~\ref{fig:reticula-pipeline},
centrándonos en el procesamiento de imagen y la extracción de información geométrica relevante con técnicas clásicas de visión computacional.


\subsubsection{Área de interés:}
\noindent
Teniendo en cuenta que la cámara del vehículo se encuentra ubicada en la parte delantera a una altura conocida y en un ángulo paralelo al suelo, el área de interés de la imagen donde se encuentran las líneas de los cajones de estacionamiento quedará siempre por debajo del horizonte de la imagen.
Por lo tanto, se puede eliminar la parte superior de la imagen para
reducir el ruido y mejorar la detección de las líneas, como se muestra en la figura \ref{fig:roi}. \\
\begin{figure}[!ht]
    \centering
    \includegraphics[width=0.9\textwidth]{img/reticule/horizont}
    \caption{Área de interés de la imagen}
    \label{fig:roi}
\end{figure}

\subsubsection{Umbralización:}
\noindent
Al área de interés de la imagen se le aplica una umbralización para realzar las líneas blancas de los cajones de estacionamiento
y eliminar otros elementos no relevantes que puedan interferir en la detección.
En la figura \ref{fig:threshold} se muestra un ejemplo de la imagen umbralizada.
\begin{figure}[!ht]
    \centering
    \includegraphics[width=0.9\textwidth]{img/reticule/thresholded}
    \caption{Imagen umbralizada}
    \label{fig:threshold}
\end{figure}

\subsubsection{Detección de contornos (Canny):}
\noindent
Se utiliza el algoritmo de Canny \cite{canny1986edge} (véase Sección~\ref{sec:canny-hough}) para detectar los bordes de las líneas en la imagen umbralizada.
En la figura \ref{fig:edges} se muestra un ejemplo de la imagen con los bordes detectados.
\begin{figure}[!ht]
    \centering
    \includegraphics[width=0.9  \textwidth]{img/reticule/canny}
    \caption{Detección de bordes mediante el algoritmo de Canny}
    \label{fig:edges}
\end{figure}

\subsubsection{Detección de líneas (Hough):}
\noindent
Se aplica la transformada de Hough \cite{ballard1981hough} (Sección~\ref{sec:canny-hough}) para detectar las coordenadas de inicio y fin de las líneas en la imagen.
En las figuras \ref{fig:hough} y \ref{fig:lines} se muestran las líneas detectadas sin fondo y sobre la imagen original, respectivamente.
\begin{figure}[!ht]
    \begin{subfigure}{0.5\textwidth}
        \includegraphics[width=\textwidth]{img/reticule/hough2}
        \caption{Líneas detectadas con la transformada de Hough}
        \label{fig:hough}
    \end{subfigure}
    \begin{subfigure}{0.5\textwidth}
        \includegraphics[width=\textwidth]{img/reticule/hough}
        \caption{Líneas detectadas en la imagen original}
        \label{fig:lines}
    \end{subfigure}
\end{figure}

\subsubsection{Representación de las ecuaciones de las líneas:}
\noindent
Una vez obtenidas las coordenadas de inicio y fin de cada línea paralela, se puede utilizar la ecuación general de la recta (véase Sección~\ref{sec:rectas-svd}):
\begin{equation}
    Ax + By + C = 0
\end{equation}
Esta ecuación permite determinar la orientación de cada línea.
Dado que las coordenadas iniciales y finales de cada línea corresponden a los valores de $x$ y $y$, respectivamente,
estas se pueden emplear para formular un sistema de ecuaciones que describa los parámetros de la recta $[A, B, C]$.
Dicho sistema puede representarse de manera matricial como sigue:
\begin{equation}
    \begin{aligned}
        \left[\begin{array}{ccc}
                      x_1 & y_1 & 1 \\
                      x_2 & y_2 & 1
                  \end{array}\right]
        \begin{bmatrix}
            A \\
            B \\
            C
        \end{bmatrix}
        =
        \begin{bmatrix}
            0 \\
            0
        \end{bmatrix}
    \end{aligned}
\end{equation}
\noindent
Esta representación permite calcular de forma precisa los coeficientes de la ecuación de la recta para cada línea detectada,
lo cual es fundamental para analizar su orientación y posición dentro de la retícula de estacionamiento.

\subsubsection{Cálculo de ecuaciones de las líneas (SVD):}
\noindent
Para calcular los coeficientes $[A, B, C]$ de las ecuaciones de las líneas detectadas, se utiliza el concepto de espacio nulo (\emph{null space}) y SVD (Sección~\ref{sec:rectas-svd}).
Este enfoque se basa en el hecho de que cualquier vector en el espacio nulo de una matriz $\mathbf{M}$ satisface la ecuación
\begin{equation}
    \mathbf{Mv} = 0
\end{equation}

\noindent
Cada línea se representa mediante dos puntos $(x_1, y_1)$ y $(x_2, y_2)$. A partir de estas coordenadas homogéneas,
se construye una matriz $\mathbf{M}$ de la forma:
\[
    \mathbf{M} = \begin{bmatrix}
        x_1 & y_1 & 1 \\
        x_2 & y_2 & 1
    \end{bmatrix}
\]
Esta matriz define el sistema de ecuaciones que describe la recta que pasa por los puntos dados.
El espacio nulo de $\mathbf{M}$ corresponde al conjunto de vectores $[A,B,C]$ que satisfacen
\[
    \mathbf{M} \cdot \begin{bmatrix}
        A \\ B \\ C
    \end{bmatrix} = \mathbf{0}.
\]
Se utiliza la Descomposición en Valores Singulares (SVD) \cite{golub2013matrix} para calcular este espacio nulo, ya que es una herramienta robusta y numéricamente estable.
La SVD descompone la matriz \(M\) en tres matrices \(U\), \(S\) y \(V\), donde el espacio nulo de \(M\) se puede obtener a partir de la última columna de la matriz \(V\), que corresponde al vector singular más pequeño (el más cercano a cero).
El vector resultante del espacio nulo se normaliza para que tenga una magnitud manejable.
Esto asegura que los coeficientes \(A\), \(B\) y \(C\) sean comparables entre distintas líneas.

\subsubsection{Cálculo de intersecciones (Clustering):}
\noindent
Una vez que se tienen las ecuaciones de todas las líneas paralelas en el plano de la cámara, se pueden calcular las intersecciones de estas líneas realizando un producto cruz entre las ecuaciones homogéneas [\(A\),\(B\),\(C\)] de todos los pares de líneas.
\noindent
El resultado de este producto cruz es la coordenada homogénea de un punto en el espacio que corresponde a la intersección de las líneas.
Si este punto es finito (cuando la tercera componente no es cero), se puede deshomogeneizar para obtener las coordenadas cartesianas en el plano de la cámara.
En cambio, si el punto es infinito (cuando la tercera componente es muy cercana a cero), significa que las líneas son paralelas y se intersectan en el infinito.

%https://scikit-learn.org/stable/modules/clustering.html#hierarchical-clustering
\noindent
Analizando los puntos de intersección obtenidos que se encuentran en el plano de la cámara, se pueden agrupar para determinar dónde está
concentrada la mayor cantidad de intersecciones.
Este punto de concentración de intersecciones debe corresponder al punto de fuga principal de la retícula de estacionamiento.
En la figura \ref{fig:intersections} se muestra un ejemplo de las intersecciones detectadas en la imagen original,
donde cada color representa un cluster diferente y el símbolo \texttt{+} blanco indica el centro de cada cluster.
\\

\begin{figure}[!ht]
    \centering
    \includegraphics[width=0.9\textwidth]{img/reticule/svd-km}
    \caption{Agrupacion de intersecciones de las líneas detectadas}
    \label{fig:intersections}
\end{figure}
\noindent
Como se discute en la Sección~\ref{sec:intersections-clustering}, para estimar la ubicación de este punto de fuga principal no es necesario tener en cuenta todas las intersecciones detectadas, sino solo aquellas que se encuentran en una zona cercana al horizonte de la imagen (Sección~\ref{sec:vanishing-points}).
Para determinar las intersecciones relevantes cercanas al horizonte, se puede definir un umbral de cercanía en la imagen que se puede ajustar experimentalmente.
Por ejemplo, en la siguiente imagen se muestran las intersecciones detectadas en la imagen original con puntos azules
y las intersecciones relevantes con un umbral de 10 píxeles con puntos amarillos.
En la figura \ref{fig:relevantInter} se muestra el resultado de la selección de las intersecciones relevantes. \\
\begin{figure}[!ht]
    \centering
    \includegraphics[width=0.9\textwidth]{img/reticule/relevantInter}
    \caption{Intersecciones detectadas en la imagen original}
    \label{fig:relevantInter}
\end{figure}

\noindent
Para agrupar las intersecciones relevantes empleamos \texttt{AgglomerativeClustering} de \texttt{scikit-learn}
(véase Sección~\ref{sec:sklearn-agglomerative}). En nuestra configuración práctica,
dejamos que el algoritmo determine el número de clusters fijando \texttt{distance\_threshold} (distancia máxima entre puntos del mismo
cluster), cuyo valor se ajusta experimentalmente al escenario. A continuación se ilustra el resultado sobre las mismas
intersecciones relevantes del ejemplo anterior: se forman 3 clusters (colores distintos) y se marca con \texttt{+} blanco el centroide de cada uno
(Fig.~\ref{fig:clusters}).
\begin{figure}[!ht]
    \centering
    \includegraphics[width=0.9\textwidth]{img/reticule/AgglomerativeClustering}
    \caption{Intersecciones agrupadas en clusters}
    \label{fig:clusters}
\end{figure}



\subsubsection{Selección del punto de fuga principal:}
\noindent
Para estimar la posición del punto de fuga principal, se puede seleccionar el cluster con mayor cantidad de intersecciones y
utilizar las líneas que generaron estos puntos para calcular la intersección de estas líneas.
La intersección de $n$ líneas (bajo un criterio de mínimos cuadrados) está dada por
el eigenvector asociado al eigenvalor más pequeño de la matriz $M$, donde:
\[
    M = \sum_{i=1}^{n} w_i l_i l_i^T
\]
Aquí, $w_i$ es un peso asociado a la línea $l_i$ y $l_i$ es la representación homogénea de la línea $i$ \cite{kanatani1998statistical}.
De esta forma, podemos conocer la posición del punto de fuga principal en el plano de la cámara.
En la figura \ref{fig:vanishingPoint} se muestra el resultado de esta estimación utilizando las líneas del cluster más grande, donde se
representa el punto de fuga principal con un símbolo \texttt{+} amarillo.
\begin{figure}[!ht]
    \centering
    \includegraphics[width=0.9\textwidth]{img/reticule/vanishingPoint}
    \caption{Punto de fuga principal estimado}
    \label{fig:vanishingPoint}
\end{figure}

\subsubsection{Selección del 2do punto de fuga principal:}
\noindent
Una vez estimado el primer punto de fuga \(\mathbf{v}_1=(x_1,y_1)\), el segundo (\(\mathbf{v}_2\)) puede obtenerse
aprovechando que corresponde a una dirección ortogonal en 3D. Con un modelo pinhole simple, pixeles cuadrados, punto principal
en el centro \(C_x\), los puntos de fuga de direcciones ortogonales verifican la relación
\( (x_1-C_x)(x_2-C_x) + f^2 \approx 0 \). De ahí se despeja la coordenada horizontal de \(\mathbf{v}_2\):
\begin{equation}
    x_2 = C_x - \frac{f^2}{\,x_1 - C_x\,}, \qquad y_2 = y_1.
\end{equation}
Aquí \(f\) es la longitud focal en píxeles, que aproximamos a partir del FOV horizontal y el ancho de imagen \(W\):
\( f = \dfrac{W}{2\,\tan(\tfrac{\text{FOV}}{2})} \). En nuestro caso, con \(\text{FOV}=90^\circ\) y \(W=1920\), se fija
\(C_x=W/2\) y se calcula \(x_2\) como en la ecuación anterior, manteniendo \(y_2=y_1\) por pertenecer ambos a la misma línea del horizonte.

\subsubsection{Filtrado de intersecciones y líneas relevantes:}
\noindent
Una vez que se han identificado los puntos de fuga principales, se puede proceder a filtrar las intersecciones y las líneas que son
relevantes para la retícula de estacionamiento.
Primero, se filtran las intersecciones que están cerca de los puntos de fuga principales, utilizando un umbral de distancia que se puede ajustar experimentalmente.
Luego, se filtran las líneas que pasan por estas intersecciones relevantes.
Este proceso ayuda a eliminar el ruido y las líneas que no contribuyen a la formación de la retícula de estacionamiento.


\subsubsection{Mediciones en la retícula detectada:}
Pendiente de redacción.


\subsection{Ajuste de la retícula mediante RANSAC}\label{sec:metodo-ransac}

Con base en los principios expuestos en la sección \ref{sec:ransac-teorico},
aplicamos una adaptación de RANSAC para estimar una homografía que haga coherente una retícula ideal con
las líneas de los cajones visibles en la imagen. Este enfoque conserva la robustez frente a outliers y
oclusiones propias del escenario de estacionamiento.

\subsection{Variación de RANSAC para ajuste de retícula:}

La variación propuesta sigue el ciclo de la Figura~\ref{fig:ramsac-flujo}:

\begin{figure}[!ht]
    \centering
    \includegraphics[width=0.99\textwidth]{img/3-metodo/ramsac-loop1.png}
    \caption{Esquema del ciclo RANSAC propuesto para el ajuste de la retícula.}
    \label{fig:ramsac-flujo}
\end{figure}


Cada iteración del ciclo comprende los siguientes pasos:
\begin{enumerate}
    \item A partir de dos puntos de fuga estimados, agrupar las líneas detectadas en dos conjuntos según su punto de fuga asociado.
    \item Muestrear aleatoriamente dos líneas de cada conjunto y calcular las cuatro intersecciones \(P_1, P_2, P_3, P_4\) (producto cruzado de pares de rectas), que definen un candidato a cajón.
    \item Estimar la homografía \(\mathbf{H}\) que mapea el cuadrado canónico de lado 1 al cuadrilátero \(\{P_k\}_{k=1}^4\).
    \item Proyectar una retícula \(n\times n\) mediante \(\mathbf{H}\) sobre la imagen.
    \item Evaluar la consistencia entre la retícula proyectada y las líneas reales (\emph{e.g.}, concordancia de orientación y proximidad en las regiones de soporte) y registrar una medida de ajuste.
    \item Almacenar temporalmente \(\mathbf{H}\) y su medida de ajuste; repetir desde el paso 2 durante \(N\) iteraciones.
\end{enumerate}


Tras completar las iteraciones, se selecciona la homografía con mejor medida de consistencia 
como representación de la retícula en la imagen. 
Esta homografía permite extender el patrón más allá del campo visible, 
infiriendo la ubicación de cajones en puntos ciegos o parcialmente ocultos, 
lo que favorece la planificación de maniobras y el estacionamiento automático. 
Además, el esquema es robusto a líneas espurias y a oclusiones, 
y tiende a ser estable entre cuadros consecutivos al reutilizar la mejor homografía previa como
punto de partida (warm-start) . En la Figura~\ref{fig:ramsac-transform} se ilustra 
la transformación de un cuadrado canónico a un cajón detectado en la imagen mediante la homografía estimada.


\begin{figure}[!ht]
    \centering
    \includegraphics[width=0.99\textwidth]{img/3-metodo/transformacion.png}
    \caption{Estimación de una homografía \(\mathbf{H}\) que mapea un cuadrado canónico (izquierda) a un cuadrilátero de un cajón detectado en la imagen (derecha), permitiendo proyectar una retícula coherente aun cuando ciertas marcas no sean visibles.}
    \label{fig:ramsac-transform}
\end{figure}

\subsection{Extracción de coordenadas en la retícula detectada}

Una vez estimada la homografía óptima \(\mathbf{H}\) mediante el esquema RANSAC descrito, 
es posible extraer información métrica y realizar mediciones en el espacio 3D del estacionamiento. 
La homografía establece una correspondencia entre puntos del plano de la retícula ideal 
(coordenadas normalizadas 2D) y puntos en el plano del suelo del mundo 3D.

Dados dos puntos en coordenadas homogéneas de la retícula normalizada:
\begin{equation}
\mathbf{p}_1 = [x_1, y_1, 1]^T, \quad \mathbf{p}_2 = [x_2, y_2, 1]^T
\end{equation}
sus correspondientes en el plano del suelo se obtienen aplicando la homografía:
\begin{equation}
\mathbf{P}_1 = \mathbf{H} \cdot \mathbf{p}_1, \quad \mathbf{P}_2 = \mathbf{H} \cdot \mathbf{p}_2
\end{equation}

Tras deshomogeneizar (dividir por la tercera coordenada), se obtienen las coordenadas 
métricas $(X_1, Y_1)$ y $(X_2, Y_2)$ en el plano del suelo. Con estas coordenadas, 
es posible calcular distancias euclidianas reales:
\begin{equation}
d = \sqrt{(X_2 - X_1)^2 + (Y_2 - Y_1)^2}
\end{equation}

Este procedimiento permite realizar mediciones precisas entre puntos de interés en la retícula, 
como las esquinas de los cajones, centros de cajones, o distancias relativas del vehículo 
respecto a la estructura del estacionamiento. La Figura~\ref{fig:distances-grid} muestra 
un ejemplo de varias distancias medidas en el espacio 3D a partir de la retícula detectada, 
donde se pueden observar las distancias desde la posición del vehículo (representado por 
el rectángulo azul en el origen) hacia diferentes puntos de la retícula proyectada.

\begin{figure}[!ht]
    \centering
    \includegraphics[width=0.6\textwidth]{img/3-metodo/distances.png}
    \caption{Ejemplo de mediciones de distancias en el espacio 3D a partir de la retícula detectada.}
    \label{fig:distances-grid}
\end{figure}


%\subsection{Representación de la posición relativa al estacionamiento}
%\noindent
Para representar la posición relativa del vehículo con respecto al espacio de estacionamiento,
se propone un sistema de coordenadas cilíndricas en el cual el origen se encuentre en la cámara del vehículo.
Esta elección provee un marco de referencia que facilita la medición de las distancias
y ángulos necesarios para maniobrar adecuadamente durante el estacionamiento.\\
\noindent
Utilizando coordenadas cilíndricas, es posible describir la posición del vehículo en términos de distancia radial, ángulo y altura,
con el origen en la cámara del vehículo permite representar de manera directa
las relaciones espaciales entre el vehículo y los límites del espacio de estacionamiento.
La altura conocida de la cámara respecto al suelo proporciona un eje de referencia constante,
mientras que los ejes orientados según la dirección y lateralidad del vehículo facilitan la interpretación y cálculo de la orientación
y desplazamiento necesarios para estacionar correctamente.\\


\subsection{Origen del sistema de coordenadas}
Para representar el origen se establecerán tres ejes coordenados:
\begin{itemize}
    \item El eje $z$ se define como la altura conocida de la cámara con respecto al suelo.
    \item El eje $x$ se define como la orientación del vehículo.
    \item El eje $y$ se define como la orientación lateral del vehículo.
\end{itemize}

\begin{figure}[!ht]
    \centering
    \begin{subfigure}{0.4\textwidth}
        \includegraphics[width=\textwidth]{img/distances_ubi_11}\label {fig:distances11}
    \end{subfigure}
    \begin{subfigure}{0.4\textwidth}
        \includegraphics[width=\textwidth]{img/distances_ubi_12}\label {fig:distances12}
    \end{subfigure}
    \caption{Origen del sistema de coordenadas.}
    \label{fig:coord}
\end{figure}
\clearpage

\subsection{Puntos de referencia}
Para representar la posición del vehículo con respecto al espacio de estacionamiento, se establecerán cuatro puntos de referencia:
\begin{itemize}
    \item El punto $1$ se define como la esquina superior izquierda del espacio de estacionamiento.
    \item El punto $2$ se define como la esquina superior derecha del espacio de estacionamiento.
    \item El punto $3$ se define como la esquina inferior izquierda del espacio de estacionamiento.
    \item El punto $4$ se define como la esquina inferior derecha del espacio de estacionamiento.
\end{itemize}
\begin{figure}[!ht]
    \centering
    \begin{subfigure}{0.4\textwidth}
        \includegraphics[width=\textwidth]{img/distances_ubi_21}\label {fig:distances21}
    \end{subfigure}
    \begin{subfigure}{0.4\textwidth}
        \includegraphics[width=\textwidth]{img/distances_ubi_22}\label {fig:distances22}
    \end{subfigure}
    \caption{Puntos de referencia.}
    \label{fig:coord2}
\end{figure}
\clearpage

\subsection{Coordenadas cilíndricas}
\noindent
Las coordenadas cilíndricas se definen como $(\rho, \theta, z)$, donde:
\begin{itemize}
    \item $\rho$ es la distancia entre el vehículo y un punto de referencia en el espacio de estacionamiento.
    \item $\theta$ es el ángulo de orientación del vehículo con respecto a un punto de referencia en el espacio de estacionamiento.
    \item $z$ es la altura de la cámara con respecto al suelo.
\end{itemize}
\begin{figure}[!ht]
    \centering
    \begin{subfigure}{0.4\textwidth}
        \includegraphics[width=\textwidth]{img/distances_ubi_31}\label {fig:distances31}
    \end{subfigure}
    \begin{subfigure}{0.4\textwidth}
        \includegraphics[width=\textwidth]{img/distances_ubi_32}\label {fig:distances32}
    \end{subfigure}
    \caption{Sistema de coordenadas cilíndricas.}
    \label{fig:coord3}
\end{figure}

\subsection{Posición relativa}
La posición relativa del vehículo con respecto al espacio de estacionamiento se representará mediante los 4 vectores de coordenadas cilíndricas:\\
\begin{center}
    $(\rho_1, \theta_1, z)$ , $(\rho_2, \theta_2, z)$ , $(\rho_3, \theta_3, z)$ , $(\rho_4, \theta_4, z)$ \\

\end{center}






% Capítulo 3: Resultados
\clearpage
\section{Resultados}
Pendiente de redacción.
\subsection{Retículas obtenidas para diferentes escenarios}
\subsection{Aplicación en algoritmos de aprendizaje por refuerzo}
\subsubsection{Highway-env}
\subsubsection{Highway-env aplicando la retícula detectada}
\subsubsection{comparación de resultados}

% Capítulo 4: Conclusiones y trabajo futuro
\clearpage
\section{Conclusiones y trabajo futuro}
Pendiente de redacción.

% Referencias
\clearpage
\bibliographystyle{acm}
\bibliography{references}

\end{document}